\section{实现完美转发}
为了避免重载具有不同值类型的参数的函数,C++引入了一种机制来实现完美转发。需要三样东西:

\begin{enumerate}
	\item 将调用形参作为纯rvalue引用(使用\&\&声明,但不使用\cppinline{const}或\cppinline{volatile})。
	\item 形参的类型必须是函数模板的形参。
	\item 将形参转发给另一个函数时,可以使用\cppinline{std::forward<>()}的辅助函数,该函数在<utility>头文件中声明。
\end{enumerate}

函数的完美转发参数,如下所示:

\begin{cppcode}
template<typename T>
void callFoo(T&& arg) {
	foo(std::forward<T>(arg)); // equivalent to foo(std::move(arg)) for passed rvalues
}
\end{cppcode}

\cppinline{std::forward<>()}实际上是一个条件性\cppinline{std::move()},这样就得到了与上面\cppinline{callFoo()}的三个(或四个)重载等价的行为:

\begin{itemize}
	\item 如果传递rvalue给\cppinline{arg},就会产生与调用\cppinline{foo(std::move(arg))}相同的效果。
	\item 如果我们传递一个lvalue给\cppinline{arg},就会产生与调用\cppinline{foo(arg)}相同的效果。
\end{itemize}

同样,可以完美传递两个参数:

\begin{cppcode}
template<typename T1, typename T2>
void callFoo(T1&& arg1, T2&& arg2) {
	foo(std::forward<T1>(arg1), std::forward<T2>(arg2));
}
\end{cppcode}

也可以\cppinline{将std::forward<>()}应用于可变参的实参,完美地将进行转发:

\begin{cppcode}
template<typename... Ts>
void callFoo(Ts&&... args) {
	foo(std::forward<Ts>(args)...);
}
\end{cppcode}

注意,不会对所有参数一次性使用forward<>(),而是会分别为每个参数使用。因此,必须将省略号("…")放在\cppinline{forward()}表达式的末尾,而不是直接放在\cppinline{args}的后面。

然而,这里到底发生了什么,需要详细的解释。

\subsection{通用(或转发)引用}

首先,将\cppinline{arg}声明为rvalue引用形参:

\begin{cppcode}
template<typename T>
void callFoo(T&& arg); // arg is universal/forwarding reference
\end{cppcode}

这可能会给人一种应该适用rvalue引用的一般规则的感觉。然而,事实并非如此。函数模板形参的rvalue引用(未限定为\cppinline{const}或\cppinline{volatile}),不遵循普通rvalue引用的规则。所以,不是一回事

\subsubsection{两个术语:通用引用和转发引用}

这样的引用称为通用引用。但C++标准中还使用了另一个术语:转发引用。这两个术语没有区别,只是一个历史问题,两个术语的含义是相同的。

这两个术语都描述了通用引用/转发引用的基本面:

\begin{itemize}
	\item 可以统一绑定到所有类型的对象(\cppinline{const}和非\cppinline{const})和值类别。
	\item 通常用来转发参数,但这并不是唯一的用法(这也是我更喜欢“通用引用”的原因之一)。
\end{itemize}

\subsubsection{通用引用绑定到所有值类别}

通用引用的重要特性是,可以绑定到任何值类别的对象和表达式:

\begin{cppcode}
template<typename T>
void callFoo(T&& arg); // arg is a universal/forwarding reference

X v;
const X c;
callFoo(v); // OK
callFoo(c); // OK
callFoo(X{}); // OK
callFoo(std::move(v)); // OK
callFoo(std::move(c)); // OK
\end{cppcode}

此外,保持所绑定对象的常量和值类别(无论我们有rvalue还是lvalue):

\begin{cppcode}
template<typename T>
void callFoo(T&& arg); // arg is a universal/forwarding reference

X v;
const X c;
callFoo(v); // OK, arg is X&
callFoo(c); // OK, arg is const X&
callFoo(X{}); // OK, arg is X&&
callFoo(std::move(v)); // OK, arg is X&&
callFoo(std::move(c)); // OK, arg is const X&&
\end{cppcode}

按照规则,类型T\&\&是\cppinline{arg}的类型

\begin{itemize}
	\item 如果引用lvalue,则为lvalue引用
	\item 如果引用rvalue,则为rvalue引用
\end{itemize}

注意,用\cppinline{const}(或\cppinline{volatile})限定的泛型rvalue引用不是通用引用。只能传递rvalue:

\begin{cppcode}
template<typename T>
void callFoo(const T&& arg); // arg is not a universal/forwarding reference

const X c;
callFoo(c); // ERROR: c is not an rvalue
callFoo(std::move(c)); // OK, arg is const X&&
\end{cppcode}

这里,还没有谈到T的类型。稍后将解释什么样的类型可以导出为通用引用的T。

稍后,将讨论使用Lambda的相应示例。

\subsection{\cppinline{std::forward<>()}}

\cppinline{callFoo()}内部,使用如下所示的通用引用:

\begin{cppcode}
template<typename T>
void callFoo(T&& arg) {
	foo(std::forward<T>(arg)); // becomes foo(std::move(arg)) for passed rvalues
}
\end{cppcode}

和\cppinline{std::move()}一样,\cppinline{std::forward<>()}也定义在头文件<utility>中。

\cppinline{std::forward<t>(arg)}其实是这样实现的:

\begin{itemize}
	\item 如果传递给函数的是T类型的rvalue,则表达式等价于\cppinline{std::move(arg)}。
	\item 如果传递给函数的是T类型的lvalue,则表达式等价于arg。
\end{itemize}

也就是说,\cppinline{std::forward<>()}是\cppinline{std::move()},仅用于传递rvalue。

就像\cppinline{std::move()}一样,\cppinline{std::forward<>()}的语义是在这里不再需要这个值,另外保留了要传递的通用引用绑定的对象类型(包括常量)和值类别。你可以争辩说,需要达成条件才不再需要这个值,但是因为不知道\cppinline{std::forward<>()}是否变成了\cppinline{std::move()},所以假设对象之后有值就是错误的。因此,使用\cppinline{std::forward<>()}之后,对象通常有效,但可能不知道具体值。

\subsubsection{\cppinline{std::forward<>()}用于成员函数}

注意,可以在调用成员函数时使用\cppinline{std::forward<>()}作为通用引用。记住,成员函数可能使用引用限定符对移动语义有特定的重载。如果不再需要该对象的值,可以使用\cppinline{std::forward<>()}来调用成员函数。

例如,假设重载了getter来提高返回临时人员名称的性能:

\begin{cppcode}
class Person
{
private:
	std::string name;
public:
	...
	void print() const {
		std::cout << "print()\n";
	}

	std::string getName() && { // when we no longer need the value
		return std::move(name); // we steal and return by value
	}
	const std::string& getName() const& { // in all other cases
		return name; // we give access to the member
	}
};
\end{cppcode}

采用通用引用的函数中,可以使用\cppinline{std::forward<>()},如下所示:

\begin{cppcode}
template<typename T>
void foo(T&& x)
{
	x.print(); // OK, no need to forward the passed value category

	x.getName(); // calls getName() const&
	std::forward<T>(x).getName(); // calls getName() && for rvalues (OK, no longer need x)
}
\end{cppcode}

使用\cppinline{std::forward<>()}之后,\cppinline{x}处于有效但未指定的状态。无论何时使用\cppinline{std::forward<>()},请确保不再使用该对象。

\subsection{完美转发的效果}

结合声明通用引用的行为和\cppinline{std::forward<>()}的使用,得到了以下行为:

\begin{cppcode}
void foo(const X&); // for constant values (read-only access)
void foo(X&); // for variable values (out parameters)
void foo(X&&); // for values that are no longer used (move semantics)
template<typename T>

void callFoo(T&& arg) { // arg is a universal/forwarding reference
	foo(std::forward<T>(arg)); // becomes foo(std::move(arg)) for passed rvalues
}

X v;
const X c;

callFoo(v); // OK, expands to foo(arg), so it calls foo(X&)
callFoo(c); // OK, expands to foo(arg), so it calls foo(const X&)
callFoo(X{}); // OK, expands to foo(std::move(arg)), so it calls foo(X&&)
callFoo(std::move(v)); // OK, expands to foo(std::move(arg)), so it calls foo(X&&)
callFoo(std::move(c)); // OK, expands to foo(std::move(arg)), so it calls foo(const X&)
\end{cppcode}

传递给callFoo()的任何参数都会变成lvalue(因为参数\cppinline{arg}是有名称的对象)。然而,\cppinline{arg}的类型取决于传递的内容:

\begin{itemize}
	\item 如果传递lvalue,\cppinline{arg}就是一个lvalue引用(传递非\cppinline{const} X时是X\&,传递\cppinline{const} X时是X\&)。
	\item 如果传递rvalue(未命名的临时对象或用\cppinline{std::move()}标记的对象),则\cppinline{arg}是rvalue引用(X\&\&或const X\&\&)。
\end{itemize}

当有rvalue引用时(即,\cppinline{arg}绑定到右值),通过\cppinline{std::forward<>()},就可以用\cppinline{std::move()}转发形参。




















