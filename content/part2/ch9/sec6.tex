\section{总结}
\begin{itemize}
	\item 带有两个\&符(\&\&)的声明可以是两个不同的类型:
	\begin{itemize}
		\item[-] 如果不是函数模板形参,它是一个普通的ravlue引用,只绑定到ravlue。
		\item[-] 如果是函数模板参数,它是一个通用引用,可以绑定到所有值类别。
	\end{itemize}
	\item 通用引用(在C++标准中称为转发引用)是一种可以通用地引用任何类型和值类别对象的引用。类型是:
		\begin{itemize}
		\item[-] lvalue引用(类型\&):绑定到lvalue
		\item[-] rvalue引用(类型\&\&):绑定到rvalue
	\end{itemize}
	\item 要完美地传递传递的实参,请使用\textit{std::forward<>()},并将该形参声明为函数模板形参的通用引用。
	\item \textit{std::forward<>()}是一个条件\textit{std::move()}。如果参数是rvalue,则扩展为\textit{std::move()}。
	\item 使用\textit{std::forward<>()}标记对象可能是有意义的,即使在调用成员函数时也是如此。
	\item 通用引用是所有重载解析的次优选择。
	\item 不要为通用引用实现泛型构造函数(或为特定类型进行约束)。
\end{itemize}


