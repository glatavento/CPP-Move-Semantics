\section{完美转发的动机}
知道了移动语义不能自动传递,所以对泛型代码会有影响。

\subsection{为什么需要完美转发}

要将带有移动语义的对象转发给函数,不仅需要绑定到rvalue引用,还需要再次使用 \cppinline{std::move()} 将其移动语义转发给另一个函数。

例如,引用重载函数的规则:

\begin{cppcode}
class X {
	...
};

// forward declarations:
void foo(const X&); // for constant values (read-only access)
void foo(X&); // for variable values (out parameters)
void foo(X&&); // for values that are no longer used (move semantics)
\end{cppcode}

调用这些函数时,有以下规则:

\begin{cppcode}
X v;
const X c;

foo(v); // calls foo(X&)
foo(c); // calls foo(const X&)
foo(X{}); // calls foo(X&&)
foo(std::move(v)); // calls foo(X&&)
foo(std::move(c)); // calls foo(const X&)
\end{cppcode}

假设通过协助函数 \cppinline{callFoo()} 间接地调用相同的参数 \cppinline{foo()}。函数还需要三个重载:

\begin{cppcode}
void callFoo(const X& arg) { // arg binds to all const objects
	foo(arg); // calls foo(const X&)
}
void callFoo(X& arg) { // arg binds to lvalues
	foo(arg); // calls foo(X&)
}
void callFoo(X&& arg) { // arg binds to rvalues
	foo(std::move(arg)); // needs std::move() to call foo(X&&)
}
\end{cppcode}

这里,\cppinline{arg} 都用作lvalue(具有名称的对象)。第一个版本将其作为 \cppinline{const} 对象转发,但其他两种情况实现了转发非 \cppinline{const} 参数的两种不同方式:

\begin{itemize}
	\item 声明为lvalue引用(绑定到没有移动语义的对象)的参数按原样传递。
	\item 声明为rvalue引用(绑定到具有移动语义的对象)的参数通过 \cppinline{std::move()} 传递。
\end{itemize}

这可以完美地转发移动语义:对于任何通过移动语义传递的参数,保持移动语义。当遇到没有移动语义的参数时,不添加移动语义。

看下 \cppinline{callFoo()} 如何调用不同的 \cppinline{foo()}:

\begin{cppcode}
X v;
const X c;
callFoo(v); // calls foo(X&)
callFoo(c); // calls foo(const X&)
callFoo(X{}); // calls foo(X&&)
callFoo(std::move(v)); // calls foo(X&&)
callFoo(std::move(c)); // calls foo(const X&)
\end{cppcode}

请记住,传递给rlvalue引用的rvalue在使用时成为lvalue,需要 \cppinline{std::move()} 再次将其作为rvalue传递。但是,有些地方不能使用 \cppinline{std::move()} 将调用 \cppinline{std::move()} 的重载实现来获取rvalue引用。

为了在泛型代码中实现完美转发,需要为每个参数进行重载。为了支持所有组合,对2个泛型参数有9个重载,对3个泛型参数有27个重载。

因此,C++11引入了一种方式来完美地转发给定的参数,不需要任何重载,仍然保持类型和具体值。

\subsubsection{完美转发 \cppinline{const} rvalue引用}

虽然 \cppinline{const} rvalue引用没有语义上的含义,但想要用 \cppinline{std::move()} 标记的常量对象的类型和值,还需要第四个重载:

\begin{cppcode}
void callFoo(const X&& arg) { // arg binds to const rvalues
	foo(std::move(arg)); // needs std::move() to call foo(const X&&)
}
\end{cppcode}

否则,将调用 \cppinline{foo(const X&)}。这通常没问题的,但在某些情况下,可能希望保留传递 \cppinline{const} rvalue引用的信息(例如,出于某种原因,提供了 \cppinline{const} 的重载)。

有了完美转发的特性,泛型代码就没必要为了对两个或三个参数,进行16和64次重载了。


















