\section{通用或普通rvalue引用?}
对普通rvalue引用和通用/转发引用使用相同的语法,这为对象是普通rvalue引用还是通用引用带来了一些有趣的情况。

\subsection{泛型类型成员的rvalue引用}

对模板形参的成员类型的rvalue引用不是通用引用。

例如:

\begin{cppcode}
template<typename T>
void foo(typename T::value_type&& arg); // not a universal reference
\end{cppcode}

这里有一个完整的例子:

\filename{generic/universalmem.cpp}
\begin{cppcode}
#include <iostream>
#include <string>
#include <vector>

template<typename T>
void insert(T& coll, typename T::value_type&& arg)
{
	coll.push_back(arg);
}

int main()
{
	std::vector<std::string> coll;
	...
	insert(coll, std::string{"prvalue"}); // OK
	...
	std::string str{"lvalue"};
	insert(coll, str); // ERROR: T::value_type&& is not a universal reference
	insert(coll, std::move(str)); // OK
	...
}
\end{cppcode}

\subsection{类模板中参数的rvalue引用}

对类模板的模板形参的rvalue引用不是通用引用。

例如:

\begin{cppcode}
template<typename T>
class C {
	T&& member; // member is not a universal reference
	...
	void foo(T&& arg); // arg is not a universal reference
};
\end{cppcode}

完整的例子:

\filename{generic/universalclass.cpp}
\begin{cppcode}
#include <iostream>
#include <string>
#include <vector>

template<typename T>
class Coll {
private:
	std::vector<T> values;
public:
	Coll() = default;

	// function in class template:
	void insert(T&& val) {
		values.push_back(val);
	}
};

int main()
{
	Coll<std::string> coll;
	...
	coll.insert(std::string{"prvalue"}); // OK
	std::string str{"lvalue"};
	coll.insert(str); // ERROR: T&& of Coll<T> is not a universal reference
	coll.insert(std::move(str)); // OK
	...
}
\end{cppcode}

通常,类模板中的函数不遵循函数模板规则。就是\cppinline{temploid},泛型代码在实例化类时遵循普通函数的规则。

\subsection{完全特化中参数的rvalue引用}

对函数模板的完全特化的形参的rvalue引用不是通用引用。

例如:

\begin{cppcode}
template<typename T> // primary template
void foo(T&& arg); // - arg is a universal reference
...
template<> // full specialization (for rvalues only)
void foo(std::string&& arg); // - arg is not a universal reference
\end{cppcode}

对于std::string类型的lvalue,仍然会调用主模板。要特化std::string的主模板的所有情况,必须提供第二个全特化版本:

\begin{cppcode}
template<> // full specialization (for lvalues)
void foo(std::string& arg);
\end{cppcode}

完整的例子:

\filename{generic/universalspec.cpp}
\begin{cppcode}
#include <iostream>
#include <string>
#include <vector>

// primary template taking a universal reference:
template<typename Coll, typename T>
void insert(Coll& coll, T&& arg)
{
	std::cout << "primary template for type T called\n";
	coll.push_back(arg);
}

// full specialization for rvalues of type std::string:
template<>
void insert(std::vector<std::string>& coll, std::string&& arg)
{
	std::cout << "full specialization for type std::string&& called\n";
	coll.push_back(arg);
}

// full specialization for lvalues of type std::string:
template<>
void insert(std::vector<std::string>& coll, const std::string& arg)
{
	std::cout << "full specialization for type const std::string& called\n";
	coll.push_back(arg);
}

int main()
{
	std::vector<std::string> coll;
	...
	insert(coll, std::string{"prvalue"}); // calls full specialization for rvalues
	std::string str{"lvalue"};
	insert(coll, str); // calls full specialization for lvalues
	insert(coll, std::move(str)); // calls full specialization for rvalues
	...
}
\end{cppcode}

注意,必须在类定义之外声明/定义成员函数模板的全特化:

\begin{cppcode}
template<typename T>
class Cont {
	...
	// primary template:
	template<typename U>
	void insert(U&& v) { // universal reference
		coll.push_back(std::forward<U>(v));
	}
	...
};

// full specializations for Cont<T>::insert<>():
// - have to be outside the class
// - need specializations for rvalues and lvalues

template<>
  template<>
void Cont<std::string>::insert<>(std::string&& v)
{
	coll.push_back(std::move(v));
}

template<>
  template<>
void Cont<std::string>::insert<>(const std::string& v)
{
	coll.push_back(v);
}
\end{cppcode}

















































