我们经常需要将返回值传递给另一个函数:

\begin{cppcode}
// pass return value of compute() to process():
process(compute(t)); // OK, uses perfect forwarding of returned value
\end{cppcode}

非泛型代码中,需要知道所涉及的类型。然而,泛型代码中,也希望\textit{compute()}的返回值完全传递给\textit{process()}。

好消息是:如果直接将返回值传递给另一个函数,该值会完美传递,保持其类型和值类别。不必担心移动语义(如果支持将自动使用)。

\subsection{默认完美传递的细节}

完整的例子:

\filename{generic/perfectpassing.cpp}
\begin{cppcode}
#include <iostream>
#include <string>

void process(const std::string&) {
	std::cout << "process(const std::string&)\n";
}
void process(std::string&) {
	std::cout << "process(std::string&)\n";
}
void process(std::string&&) {
	std::cout << "process(std::string&&)\n";
}

const std::string& computeConstLRef(const std::string& str) {
	return str;
}
	std::string& computeLRef(std::string& str) {
	return str;
}
	std::string&& computeRRef(std::string&& str) {
	return std::move(str);
}
	std::string computeValue(const std::string& str) {
	return str;
}

int main()
{
	process(computeConstLRef("tmp")); // calls process(const std::string&)
	
	std::string str{"lvalue"};
	process(computeLRef(str)); // calls process(std::string&)
	
	process(computeRRef("tmp")); // calls process(std::string&&)
	process(computeRRef(std::move(str))); // calls process(std::string&&)
	
	process(computeValue("tmp")); // calls process(std::string&&)
}
\end{cppcode}

\begin{itemize}
	\item 如果\textit{compute()}返回一个const lvalue引用:

\begin{cppcode}
const std::string& computeConstLRef(const std::string& str) {
	return str;
}
\end{cppcode}
返回值的值类别是lvalue,这意味着返回值会完美转发,并与\textit{const} lvalue引用匹配:

\begin{cppcode}
process(computeConstLRef("tmp")); // calls process(const std::string&)
\end{cppcode}
\item 如果\textit{compute()}返回一个非\textit{const} lvalue引用:

\begin{cppcode}
std::string& computeLRef(std::string& str) {
	return str;
}
\end{cppcode}
返回值的值类别是lvalue,这意味着返回值会完全转发,并与非\textit{const} lvalue引用的最佳匹配:

\begin{cppcode}
std::string str{"lvalue"};
process(computeLRef(str)); // calls process(std::string&)
\end{cppcode}
\item 如果\textit{compute()}返回rvalue引用:

\begin{cppcode}
std::string&& computeRRef(std::string&& str) {
	return std::move(str);
}
\end{cppcode}
返回值的值类别是xvalue,这意味着返回值会完全转发为rvalue引用,允许\textit{process()}窃取值:

\begin{cppcode}
process(computeRRef("tmp")); // calls process(std::string&&)
process(computeRRef(std::move(str))); // calls process(std::string&&)
\end{cppcode}
\item 如果\textit{compute()}按值返回临时对象:

\begin{cppcode}
std::string computeValue(const std::string& str) {
	return str;
}
\end{cppcode}
返回值的值类别是prvalue,返回值完全转发为rvalue引用,也允许\textit{process()}窃取值:

\begin{cppcode}
process(computeValue("tmp")); // calls process(std::string&&)
\end{cppcode}
\end{itemize}

注意,通过返回\textit{const}值:

\begin{cppcode}
const std::string computeConstValue(const std::string& str) {
	return str;
}
\end{cppcode}

或\textit{const} rvalue引用:

\begin{cppcode}
const std::string&& computeConstRRef(std::string&& str) {
	return std::move(str);
}
\end{cppcode}

再次禁用移动语义:

\begin{cppcode}
process(computeConstValue("tmp")); // calls process(const std::string&)
process(computeConstRRef("tmp")); // calls process(const std::string&)
\end{cppcode}

如果有\textit{const}\&\&的声明,可以接受这样的重载。

因此:不要将value返回的值标记为\textit{const},也不要将返回的非\textit{const} rvalue引用标记为\textit{const}。


