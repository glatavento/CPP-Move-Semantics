当使用具体类型的rvalue引用形参声明函数时,只能将这些形参绑定到rvalue。例如:

\begin{cppcode}
void rvFunc(std::string&&); // forward declaration

std::string s{ ... };
rvFunc(s); // ERROR: passing an lvalue to an rvalue reference
rvFunc(std::move(s)); // OK, passing an xvalue
\end{cppcode}

但请注意,有时传递lvalue是可行的。例如:

\begin{cppcode}
void rvFunc(std::string&&); // forward declaration

rvFunc("hello"); // OK, although "hello" is an lvalue
\end{cppcode}

记住,字符串文字作为表达式使用时是lvalue。因此,不能传递给rvalue引用。但是,这里涉及到一个隐藏的操作,因为实参的类型(6个常量字符的数组)与形参的类型不匹配。隐式类型转换由string构造函数执行,创建了一个没有名称的临时对象。

因此,真正的使用方式如下:

\begin{cppcode}
void rvFunc(std::string&&); // forward declaration

rvFunc(std::string{"hello"}); // OK, "hello" converted to a string is a prvalue
\end{cppcode}






























































