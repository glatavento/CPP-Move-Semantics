\section{getter的返回类型}
在C++11之前,当为复制开销很大的成员实现getter时,有以下几种选择:

\begin{itemize}
	\item 返回值
	\item 通过左值引用返回
\end{itemize}

简单地讨论一下这些方案。

\subsection{传值返回}

按值返回的getter看起来是这样的(记住:不要使用\textit{const}按值返回,否则禁用移动语义):

\begin{cppcode}
class Person
{
private:
	std::string name;
public:
	...
	std::string getName() const {
		return name;
	}
};
\end{cppcode}

这段代码是安全的,但每次获取\textit{name}时,都可能复制\textit{name}。

例如,只是检查是否有没名字的人会有很大的开销:

\begin{cppcode}
std::vector<Person> coll;
...
for (const auto& person : coll) {
	if (person.getName().empty()) { // OOPS: copies the name
		std::cout << "found empty name\n";
	}
}
\end{cppcode}

如果将此方法与返回引用的方法进行比较,可以看到按值返回字符串的版本的性能开销是引用的2到100倍(前提是名称的长度较大,这样SSO就没有帮助了)。对图像或数千个元素的集合的成员可能更糟糕。在这种情况下,getter通常返回(const)引用以提高性能。

\subsection{引用返回}

通过引用返回的getter看起来像这样:

\begin{cppcode}
class Person
{
private:
	std::string name;
public:
	...
	const std::string& getName() const {
		return name;
	}
};
\end{cppcode}

这样更快,但不安全,因为调用者必须确保返回引用的对象生命周期足够长。实际上,使用返回引用值比原始对象的时间长的话,会有生命周期上的风险。

会踩入这种陷阱的一种方式是使用基于范围的for循环,如下所示:

\begin{cppcode}
for (char c : returnPersonByValue().getName()) { // OOPS: undefined behavior
	if (c == ' ') {
		...
	}
}
\end{cppcode}

注意,循环的右边有一个函数,它返回一个我们用getter引用的临时对象。但是,定义了基于范围的for循环,使上面的代码等价于以下代码:

\begin{cppcode}
reference range = returnPersonByValue().getName();
// OOPS: returned temporary object destroyed here
for (auto pos = range.begin(), end = range.end(); pos != end; ++pos) {
	char c = *pos;
	if (c == ' ') {
		...
	}
}
\end{cppcode}

开始迭代之前,初始化\textit{reference1},必须使用传递的范围两次(一次调用begin(),一次调用end()),并希望避免创建副本(这可能代价很高,甚至不可能)。一般来说,引用会延长所引用内容的生命周期。在本例中,\textit{range}并不引用由\textit{returnPersonByValue()}返回的\textit{Person},而\textit{range}指向\textit{getName()}的返回值,是对返回的\textit{Person}的引用。因此,\textit{range}扩展了引用的生命周期。因此,在第一个语句的末尾,返回的临时对象将销毁,并且在遍历名称的字符时使用对已销毁对象名称的引用。

最好的情况下,在这里会得到一个核心转储(段错误),这样就可以看到出现了明显的错误。最坏的情况是,发布了软件,并出现了未定义行为。

如果getter按值返回名称,这样的代码就不会有问题。这样,\textit{range}将扩展名称副本的生存期,以便我们使用该名称直到\textit{range}的生存期结束。

\subsection{使用移动语义解决困境}

通过移动语义,就可以解决这个困境的方法。如果这样做安全,可以通过引用返回。如果遇到生命周期的问题,可以通过值返回。

方法如下:

\begin{cppcode}
class Person
{
private:
	std::string name;
public:
	...
	std::string getName() && { // when we no longer need the value
		return std::move(name); // we steal and return by value
	}
	const std::string& getName() const& { // in all other cases
		return name; // we give access to the member
	}
};
\end{cppcode}

用不同的引用限定符重载getter,方法与重载带有\&\&和\textit{const}\&形参的函数相同:

\begin{itemize}
	\item 带有\&\&限定符的是有不再需要值的对象时使用(一个即将死亡的对象或用\textit{std::move()}标记的对象)。
	\item 带有\textit{const}\&限定符的版本用于所有其他情况。总是合适的,但只是无法获得\&\&版本时的备选方案。因此,如果有一个不会销毁的对象或没标记为\textit{std::move()},就会使用此函数。
\end{itemize}

现在的性能和安全性都很好:

\begin{cppcode}
Person p{"Ben"};
std::cout << p.getName(); // 1) fast (returns reference)
std::cout << returnPersonByValue().getName(); // 2) fast (uses move())

std::vector<Person> coll;
...
for (const auto& person : coll) {
	if (person.getName().empty()) { // 3) fast (returns reference)
		std::cout << "found empty name\n";
	}
}

for (char c : returnPersonByValue().getName()) { // 4) safe and fast (uses move())
	if (c == ' ') {
		...
	}
}
\end{cppcode}

语句1)和3)使用\textit{const}\&的版本,因为有一个对象没有使用\textit{std::move()}标记。语句2)和4)使用\&\&的版本,因为我们为一个临时对象调用\textit{getName()}。因为临时对象即将销毁,getter可以将成员名移出作为返回值,这意味着不必为返回值分配新的内存,可以窃取了成员的值。

您可能还记得,return语句不应该使用\textit{std::move()}来返回已经销毁的局部对象。本例中,不返回任何局部对象,而是返回一个成员,其生存期不以成员函数的结束为结束。

\subsubsection{\textit{std::move()}用于成员函数中}

请注意,这个特性意味着即使在调用成员函数时也可以使用\textit{std::move()}。例如:

\begin{cppcode}
void foo()
{
	Person p{ ... };
	...
	coll.push_back(p.getName()); // calls getName() const&
	...
	coll.push_back(std::move(p).getName()); // calls getName() && (OK, p no longer used)
}
\end{cppcode}

使用\textit{getName()}时,使用\textit{std::move()}将提高程序的性能。不返回对\textit{const} std::string的引用(只能复制),返回值是作为非\textit{const}字符串返回的\textit{p}的移动名,因此\textit{push_back()}可以使用移动语义将其移动到\textit{coll}中。通常,在这之后,\textit{p}处于有效但未定义的状态。

关于在C++标准库中使用此特性的示例,请参见std::optional<>。
















