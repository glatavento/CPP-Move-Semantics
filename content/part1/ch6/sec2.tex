\section{可销毁和可转移}
我们讨论一下如何确保已移动对象满足支持分配和销毁的基本要求。

\subsection{可分配和可销毁的已移动对象}

大多数类中,生成的特殊移动成员函数对已移动的对象可以使用赋值操作符和析构函数。如果每个已移动成员都是可赋值和可销毁的,那么整个已移动对象的赋值和销毁都没问题:

\begin{itemize}
	\item 通过从相应的源成员中获取状态,赋值将覆盖该成员的未定义状态。
	\item 析构函数将销毁成员(具有未定义状态)。
\end{itemize}

通常应该从对象中移动的成员,但对象处于有效但未定义的状态,所以生成的移动操作通常只对创建成员状态正确的对象进行操作。

例如,以下类:

\begin{cppcode}
class Customer {
private:
	std::string name;
	std::vector<int> values;
	...
};
\end{cppcode}

当\textit{customer}移走时,保证\textit{name}和\textit{values}都处于有效的状态,以便析构函数(由\textit{customer}的析构函数调用)可以正常工作:

\begin{cppcode}
void foo()
{
	Customer c{"Michael Spencer"};
	...
	process(std::move(c));
	// both name and values have valid but unspecified states
	...
} // destructor of c will clean up name and values (whatever their state is)
\end{cppcode}

另外,给\textit{c}赋一个新值也可以,所以就同时赋了\textit{name}和\textit{values}。

\subsection{不可销毁的已移动对象}

这时在极少数情况下可能会出现的问题,通常实现赋值操作符或析构函数来完成对成员赋值或销毁的工作。

考虑以下类,使用固定大小的线程对象数组,其数量可变,为此在析构函数中显式调用join()来等待线程对象的结束:

\filename{basics/tasks.hpp}
\begin{cppcode}
#include <array>
#include <thread>
class Tasks {
	private:
	std::array<std::thread,10> threads; // array of threads for up to 10 tasks
	int numThreads{0}; // current number of threads/tasks
	public:
	Tasks() = default;
	// pass a new thread:
	template <typename T>
	void start(T op) {
		threads[numThreads] = std::thread{std::move(op)};
		++numThreads;
	}
	...
	// at the end wait for all started threads:
	~Tasks() {
		for (int i = 0; i < numThreads; ++i) {
			threads[i].join();
		}
	}
};
\end{cppcode}

目前为止,该类还不支持移动语义,因为用户声明了析构函数。因为禁用复制std::thread,所以也不能复制\textit{Tasks}对象。但是,可以启动多个任务并等待结束:

\filename{basics/tasks.cpp}
\begin{cppcode}
#include "tasks.hpp"
#include <iostream>
#include <chrono>
int main()
{
	Tasks ts;
	ts.start([]{
		std::this_thread::sleep_for(std::chrono::seconds{2});
		std::cout << "\nt1 done" << std::endl;
	});
	ts.start([]{
		std::cout << "\nt2 done" << std::endl;
	});
}
\end{cppcode}

现在通过生成默认的移动操作来启用移动语义
(详见\fninline{basics/tasksbug.cpp}):

\begin{cppcode}
class Tasks {
private:
	std::array<std::thread,10> threads; // array of threads for up to 10 tasks
	int numThreads{0}; // current number of threads/tasks
public:
	...
	// OOPS: enable default move semantics:
	Tasks(Tasks&&) = default;
	Tasks& operator=(Tasks&&) = default;
	// at the end wait for all started threads:
	~Tasks() {
		for (int i = 0; i < numThreads; ++i) {
			threads[i].join();
		}
	}
};
\end{cppcode}

这种情况下会遇到麻烦,因为默认生成的移动操作可能会创建无效的任务。比如:

\filename{basics/tasksbug.cpp}
\begin{cppcode}
#include "tasksbug.hpp"
#include <iostream>
#include <chrono>
#include <exception>
int main()
{
	try {
		Tasks ts;
		ts.start([]{
			std::this_thread::sleep_for(std::chrono::seconds{2});
			std::cout << "\nt1 done" << std::endl;
		});
		ts.start([]{
			std::cout << "\nt2 done" << std::endl;
		});
		// OOPS: move tasks:
		Tasks other{std::move(ts)};
	}
	catch (const std::exception& e) {
		std::cerr << "EXCEPTION: " << e.what() << std::endl;
	}
}
\end{cppcode}

开始时,通过将两个任务传递给\textit{Tasks}对象\textit{ts}来启动它们。因此,在\textit{ts}中线程数组有两个元素,\textit{numThreads}为2。但是,容器的移动操作会移动元素,因此在\textit{std::move()}之后,\textit{ts}不再包含任何运行的线程对象。因此,\textit{numThreads}只是复制而已,也就是创建了一个无效的任务。析构函数最终会对前两个任务调用join(),这会抛出一个异常(这将导致析构函数中出现致命错误)。

目前的问题是,需要两个成员一起才能定义一个有效的状态,而默认生成的移动语义会产生不一致的状态,因此会让析构函数失败。但不可能总是避免这个问题,因为对象销毁时,可能需要显式地对象元素的子集做一些操作。这样,默认生成的移动操作可能不起作用,应该禁用或修复它们。

可能的使用办法是:

\begin{itemize}
	\item 使用析构函数处理已移动状态(本例中,如果线程对象的joinable()为true,就只能调用join())
	\item 自己实现移动操作
	\item 禁用移动语义(这是特殊移动成员函数的行为)
\end{itemize}

根据规则0,应该将容易出错的资源管理封装在helper类型中,以便应用程序开发者实现0号特殊成员函数。这样,就可以使用一个助手类(模板),其既提供了成员(std::array和用于实际使用的元素数量的成员),又提供了移动操作的正确实现。

整个问题也是std::thread类设计错误的副作用,该类型不遵循RAII原则。对于所有运行线程的std::thread,必须在调用析构函数之前调用join()(或detach());否则,线程的析构函数将抛出异常。C++20中,可以使用std::jthread类,如果对象为一个正在运行的线程,会自动调用join()。




















