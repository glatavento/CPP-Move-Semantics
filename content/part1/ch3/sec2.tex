\section{实现复制/移动函数}
可以自己实现特殊的移动成员函数。这与实现复制构造函数和赋值操作符的方式大致相同。唯一的区别是形参需要声明为非\cppinline{const}右值引用,并且在实现内部必须指定在何处优化复制操作。

让我们看一个类,同时实现了特殊的复制和移动成员函数,用于在对象复制和移动时:

\filename{basics/customerimpl.hpp}
\begin{cppcode}
#include <string>
#include <vector>
#include <iostream>
#include <cassert>

class Customer {
	private:
	std::string name; // name of the customer
	std::vector<int> values; // some values of the customer
	public:
	Customer(const std::string& n)
	: name{n} {
		assert(!name.empty());
	}

	std::string getName() const {
		return name;
	}

	void addValue(int val) {
		values.push_back(val);
	}

	friend std::ostream& operator<< (std::ostream& strm, const Customer& cust) {
		strm << '[' << cust.name << ": ";
		for (int val : cust.values) {
			strm << val << ' ';
		}
		strm << ']';
		return strm;
	}

	// copy constructor (copy all members):
	Customer(const Customer& cust)
	: name{cust.name}, values{cust.values} {
		std::cout << "COPY " << cust.name << '\n';
	}

	// move constructor (move all members):
	Customer(Customer&& cust) // noexcept declaration missing
	: name{std::move(cust.name)}, values{std::move(cust.values)} {
		std::cout << "MOVE " << name << '\n';
	}

	// copy assignment (assign all members):
	Customer& operator= (const Customer& cust) {
		std::cout << "COPYASSIGN " << cust.name << '\n';
		name = cust.name;
		values = cust.values;
		return *this;
	}

	// move assignment (move all members):
	Customer& operator= (Customer&& cust) { // noexcept declaration missing
		std::cout << "MOVEASSIGN " << cust.name << '\n';
		name = std::move(cust.name);
		values = std::move(cust.values);
		return *this;
	}
};
\end{cppcode}

让我们详细看看特殊的拷贝/移动成员函数的实现。

请注意,手动实现移动构造函数和移动赋值操作符时,通常都应该有noexcept声明,这将在移动语义和noexcept章节中讨论。

\subsection{复制构造函数}

复制构造函数的实现如下:

\begin{cppcode}
class Customer {
private:
	std::string name; // name of the customer
	std::vector<int> values; // some values of the customer

public:
	...
	// copy constructor (copy all members):
	Customer(const Customer& cust)
	: name{cust.name}, values{cust.values} {
		std::cout << "COPY " << cust.name << '\n';
	}
	...
};
\end{cppcode}

自动生成的复制构造函数,只复制所有成员。实现中,我们只添加了一条打印特定\cppinline{customer}复制时的语句。

\subsection{移动构造函数}

移动构造函数的实现如下:

\begin{cppcode}
class Customer {
private:
	std::string name; // name of the customer
	std::vector<int> values; // some values of the customer
public:
	...
	// move constructor (move all members):
	Customer(Customer&& cust) // noexcept declaration missing
	: name{std::move(cust.name)}, values{std::move(cust.values)} {
		std::cout << "MOVE " << name << '\n';
	}
	...
};
\end{cppcode}

同样,这也是默认生成的移动构造函数所做的,并使用附加的打印语句进行扩展。

与复制构造函数的区别是将形参声明为非\cppinline{const}右值引用然后移动成员。

注意,这里非常重要的一点:移动语义没有传递。当使用\cppinline{cust}的成员初始化成员时,必须用\cppinline{std::move()}。没有这个,就只能复制它们(移动构造函数的性能与复制构造函数相同)。

您可能想知道为什么没有传递移动语义。我们没有声明参数\cppinline{cust}只接受带有移动语义的对象吗?但请注意这里的语义:当调用者不再需要该值时,将调用该函数。在移动构造函数内部,现在有了要处理的值,必须决定在哪里需要它,需要多长时间。特别是,可能多次需要这个值,并且不会在第一次使用时丢失。

因此,移动语义没有传递是一个特性,而不是一个bug。如果要传递移动语义,就不能使用两次传递了移动语义的对象。

例如:

\begin{cppcode}
void insertTwice(std::vector<std::string>& coll, std::string&& str)
{
	coll.push_back(str); // copy str into coll
	coll.push_back(std::move(str)); // move str into coll
}
\end{cppcode}

如果\cppinline{str}的所有使用都隐式地具有移动语义,则\cppinline{str}的值将在第一次\cppinline{push_back()}调用时移走。

这里需要学习的是,将形参声明为右值引用限制了可以传递给该函数的内容,但其行为与任何其他该类型的非\cppinline{const}对象一样。必须再次指定何时何地不再需要这个值。有关更多细节,后续正式讨论当右值变成左值时会继续。

另一个注意事项:当复制构造函数的打印语句打印传递的客户的名称时:

\begin{cppcode}
Customer(const Customer& cust)
: name{cust.name}, values{cust.values} {
	std::cout << "COPY " << cust.name << '\n'; // cust.name still there
}
\end{cppcode}

移动构造函数不能使用\cppinline{custom.name},因为在构造函数初始化时,该值可能已经移走了。必须使用新对象的成员:

\begin{cppcode}
Customer(Customer&& cust)
: name{std::move(cust.name)}, values{std::move(cust.values)} {
	std::cout << "MOVE " << name << '\n'; // have to use name (cust.name moved away)
}
\end{cppcode}

注意,应该始终使用noexcept规范来实现移动构造函数,以提高\cppinline{Customer} vector重新分配时的性能。

\subsection{复制赋值操作符}

复制赋值操作符的实现如下:

\begin{cppcode}
class Customer {
	private:
	std::string name; // name of the customer
	std::vector<int> values; // some values of the customer
	public:
	...
	// copy assignment (assign all members):
	Customer& operator= (const Customer& cust) {
		std::cout << "COPYASSIGN " << cust.name << '\n';
		name = cust.name;
		values = cust.values;
		return *this;
	}
...
};
\end{cppcode}

与自动生成的复制赋值操作符一样,只需对所有成员进行赋值。唯一的区别是开头的打印语句。

实现赋值操作符时,可以(也许应该)检查对象对自身的赋值。但请注意,生成的默认赋值操作符不会这样做,在讨论使用移动赋值操作符进行自我赋值时,请查看关于这一点的解释。

您可能还希望使用引用限定符声明赋值操作符。

\subsection{移动赋值运算符}

移动赋值操作符的实现如下:

\begin{cppcode}
class Customer {
	private:
	std::string name; // name of the customer
	std::vector<int> values; // some values of the customer
	public:
	...
	// move assignment (steal all members):
	Customer& operator= (Customer&& cust) { // noexcept declaration missing
		std::cout << "MOVEASSIGN " << cust.name << '\n';
		name = std::move(cust.name);
		values = std::move(cust.values);
		return *this;
	}
};
\end{cppcode}

同样,必须将移动构造函数声明为接受非\cppinline{const}右值引用的函数。然后在主体中,必须实现如何改进通常的复制,因为我们可以从源对象中窃取值。

本例中,我们做了自动生成的移动赋值操作符所做的事情:对函数体中的成员进行移动赋值,而不是复制赋值。此外,添加了打印语句。最后,返回对象以供进一步使用。

注意,通常应该使用noexcept规范来实现移动赋值操作符。

您可能还希望使用引用限定符声明移动赋值操作符。

\subsubsection{处理对象自身的移动分配}

可能想知道在实现移动赋值操作符时,是否应该检查对象对自身的赋值。例如,自移动可能会发生如下情况:

\begin{cppcode}
Customer c{"GNU's Not Unix"};
c = std::move(c); // move assigns c to itself
\end{cppcode}

这看起来很容易避免,可以使用引用和指针。这种情况下,两个对象是否相同并不明显。下面的代码是一个非常简单的例子:

\begin{cppcode}
std::vector<Customer> coll;
coll.emplace_back("GNU's Not Unix"); // coll has 1 element
coll[0] = std::move(coll.back()); // move assigns the only element to itself
\end{cppcode}

当对象自移动时,C++标准库中的所有类型都接收一个有效但未定义的状态。默认情况下,可能会丢失成员的值,如果类型不能正确地处理具有任意值的成员,甚至可能会遇到更严重的问题。事实上,在以下情况下,可能产生有问题的状态:

\begin{itemize}
	\item 已移动成员的一些值有问题
	\item 成员的值相互依赖
\end{itemize}

防止自赋值的传统/简单方法是检查两个操作数是否相同(具有相同的地址)。在实现移动赋值操作符时也可以这样做:

\begin{cppcode}
Customer& operator= (Customer&& cust) { // noexcept declaration missing
	std::cout << "MOVEASSIGN " << cust.name << '\n';
	if (this != &cust) { // move assignment to myself?
		name = std::move(cust.name);
		values = std::move(cust.values);
	}
	return *this;
}
\end{cppcode}

使用这样的检查非常廉价,它的好处是对象将其值保存。但请注意,在递归数据结构中,如果(移动)子对象分配给父对象,也可能有问题。如果在赋给新值之前先删除父对象(拥有子对象)的旧值,可能会赋给一个已删除的对象。这种情况下,对象并不相同,所以比较它们的地址没有什么用。

此外,考虑以下编程风格:

\begin{itemize}
	\item Scott Meyers在《Effective C++》一书中说,当使用助手类正确封装资源管理时(那时还没有移动语义),自赋值通常不是问题。此外,他指出,还必须确保操作符的异常安全。
	\item 《C++核心指南》将这个问题归类为“百万分之一的问题”,在实践中似乎并不相关。
\end{itemize}

似乎在大多数情况下,可以忽略自移动赋值,在你定义的类型中,提供了与C++标准库相同的保证:移动将一个对象赋值给它自己之后,该对象处于一个有效但未定义的状态。

\subsection{使用特殊的复制/移动成员函数}

让我们用一个小程序来测试\cppinline{Customer}类:

\filename{basics/customerimpl.cpp}
\begin{cppcode}
#include "customerimpl.hpp"
#include <iostream>
#include <string>
#include <vector>
#include <algorithm>

int main()
{
	std::vector<Customer> coll;
	for (int i=0; i<12; ++i) {
		coll.push_back(Customer{"TestCustomer " + std::to_string(i-5)});
	}

	std::cout << "---- sort():\n";
	std::sort(coll.begin(), coll.end(),
			  [] (const Customer& c1, const Customer& c2) {
			     return c1.getName() < c2.getName();
			  });
}
\end{cppcode}

\textbf{初始化}

第一部分是有12个\cppinline{Customer}的向量的初始化:

\begin{cppcode}
std::vector<Customer> coll;
for (int i=0; i<12; ++i) {
	coll.push_back(Customer{"TestCustomer " + std::to_string(i-5)});
}
\end{cppcode}

这个初始化可能有以下输出:

\begin{outputcode}
MOVE TestCustomer -5
MOVE TestCustomer -4
COPY TestCustomer -5
MOVE TestCustomer -3
COPY TestCustomer -5
COPY TestCustomer -4
MOVE TestCustomer -2
MOVE TestCustomer -1
COPY TestCustomer -5
COPY TestCustomer -4
COPY TestCustomer -3
COPY TestCustomer -2
MOVE TestCustomer 0
MOVE TestCustomer 1
MOVE TestCustomer 2
MOVE TestCustomer 3
COPY TestCustomer -5
COPY TestCustomer -4
COPY TestCustomer -3
COPY TestCustomer -2
COPY TestCustomer -1
COPY TestCustomer 0
COPY TestCustomer 1
COPY TestCustomer 2
MOVE TestCustomer 4
MOVE TestCustomer 5
MOVE TestCustomer 6
\end{outputcode}

每次插入新对象时,都会创建一个临时对象并将其移动到向量中。因此,对于每个元素,我们都有一个MOVE。

另外,有几份副本上有COPY的标记。出现副本的原因:vector会不时重新分配其内部内存(容量),以便它足够大,可以容纳所有元素。在这种情况下,vector从有存储1个元素的内存增长到有存储2个、4个、8个,最后是16个元素。因此,必须先复制1 2 4,然后是vector中已经存在的8个元素。在循环之前使用\cppinline{col.reserve(20)}可以避免这些副本的出现。但是,您可能想知道为什么这里不使用移动。这与缺少的noexcept声明有关,我们将在关于移动语义和noexcept的章节中讨论。

注意,向vector增加容量的确切策略是特定于实现的。因此,当实现增长不同时,输出可能会有所不同(例如,每次增长50\%)。

\textbf{排序}

接下来,我们按名称对所有元素进行排序:

\begin{cppcode}
std::sort(coll.begin(), coll.end(),
	[] (const Customer& c1, const Customer& c2) {
		return c1.getName() < c2.getName();
	});
\end{cppcode}

排序可能会出现以下输出:

\begin{outputcode}
MOVE TestCustomer -4
MOVEASSIGN TestCustomer -5
MOVEASSIGN TestCustomer -4
MOVE TestCustomer -3
MOVEASSIGN TestCustomer -5
MOVEASSIGN TestCustomer -4
MOVEASSIGN TestCustomer -3
MOVE TestCustomer -2
MOVEASSIGN TestCustomer -5
MOVEASSIGN TestCustomer -4
MOVEASSIGN TestCustomer -3
MOVEASSIGN TestCustomer -2
MOVE TestCustomer -1
MOVEASSIGN TestCustomer -5
MOVEASSIGN TestCustomer -4
MOVEASSIGN TestCustomer -3
MOVEASSIGN TestCustomer -2
MOVEASSIGN TestCustomer -1
MOVE TestCustomer 0
MOVEASSIGN TestCustomer 0
MOVE TestCustomer 1
MOVEASSIGN TestCustomer 1
MOVE TestCustomer 2
MOVEASSIGN TestCustomer 2
MOVE TestCustomer 3
MOVEASSIGN TestCustomer 3
MOVE TestCustomer 4
MOVEASSIGN TestCustomer 4
MOVE TestCustomer 5
MOVEASSIGN TestCustomer 5
MOVE TestCustomer 6
MOVEASSIGN TestCustomer 6
\end{outputcode}

整个排序只有元素移动,有时是创建一个新的临时对象(MOVE),有时是分配值到不同的位置(MOVEASSIGN)。

同样,输出取决于\cppinline{sort()}的确切实现,这是特定于实现的。

\subsubsection{保存的副本数量}

这个程序的输出再次证明了移动语义的好处。在支持移动语义前,我们只在插入和排序元素时有副本。然而,即使在这个有12个元素的小程序中,移动语义将44个拷贝转换成廉价的移动。对于1万名客户,我们将节省超过15万份拷贝。请注意,每个\cppinline{Customer}副本都意味着最多分配2个内存,稍后将进行相应的释放。

请再次注意:

\begin{itemize}
	\item \cppinline{std::sort()}是特定于实现的,保存的副本数量可能不同。
	\item 执行的唯一副本是由于重新分配vector用于存储元素的内存造成的。它们也应该使用移动操作,这将在关于移动语义的章节中讨论。
\end{itemize}

























