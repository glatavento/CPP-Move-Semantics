\section{三五法则}
是否自动生成或自动生成哪个特殊成员函数取决于刚才的规则,许多开发者并不知道这些规则。因此,即使在C++11之前,指导方针是不提供任何或所有用于复制、赋值和销毁的特殊成员函数。

\begin{itemize}
	\item C++11之前,这条原则称为“3法则”:要么声明全部三种(复制构造函数、赋值操作符和析构函数),要么一个都不声明。
	\item 从C++11开始,该规则就变成了“5法则”,通常的表述方式是:要么声明所有5种(复制构造函数、移动构造函数、复制赋值操作符、移动赋值操作符和析构函数),要么一个都不声明。
\end{itemize}

在这里,声明的意思是:

\begin{itemize}
	\item 要么实现 (\{...\})
	\item 或者声明为默认 (=default)
	\item 或声明为已删除 (=delete)
\end{itemize}

当其中一个特殊成员函数被实现、默认或删除时,应该同时实现、默认或删除所有其他四个特殊成员函数。

但是,您应该注意这条规则。我建议更多地把它作为一个指导方针,当其中一个特殊成员函数是用户声明的时候,仔细考虑这5个特殊成员函数。

为了只启用复制语义,应该使用默认复制特殊成员函数,而不声明特殊的移动成员函数(删除和默认特殊移动成员函数是行不通的,实现它们会使类变得复杂)。如果生成的移动语义创建了无效状态,则建议使用此选项,我们将在关于无效已移动状态的章节中讨论这一点。

当应用5规则时,有时开发者使用它来为新的移动操作添加声明,却不理解这意味着什么。开发者只是用=default来声明移动操作,因为已经实现了复制操作,所以他们希望遵循5规则。

因此,我通常教授的“5法则”或“3法则”是:

\begin{itemize}
	\item 如果声明了复制构造函数、移动构造函数、复制赋值操作符、移动赋值操作符或析构函数,则必须仔细考虑如何处理其他特殊成员函数。
	\item 如果不理解移动语义,只考虑复制构造函数、复制赋值操作符和析构函数(如果声明其中之一)。如果有疑问,可以使用=default声明复制特殊成员函数来禁用移动语义。
\end{itemize}






















































