\section{移动的对象}
在\cppinline{std::move()}之后,移动的对象不会(部分)销毁。它们仍然是有效的对象,至少会为其调用析构函数。然而,它们也是有效的,因为它们具有一致的状态,所有操作都按照预期工作,但不知道的是它们的值。这就像使用类型的参数,而不知道传递了哪个值。

\subsection{有效但定义的状态}

C++标准库保证移动的对象处于有效但未定义的状态。

考虑如下代码:

\begin{cppcode}
std::string s;
...
coll.push_back(std::move(s));
\end{cppcode}

用\cppinline{std::move()}传递\cppinline{s}之后,可以获取字符数量,打印相应的值,甚至赋一个新值。如果不先检查字符数量,就不能打印第一个字符或任何其他字符:

\begin{cppcode}
foo(std::move(s)); // keeps s in a valid but unclear state

std::cout << s << '\n'; // OK (don’t know which value is written)
std::cout << s.size() << '\n'; // OK (writes current number of characters)
std::cout << s[0] << '\n'; // ERROR (potentially undefined behavior)
std::cout << s.front() << '\n'; // ERROR (potentially undefined behavior)
s = "new value"; // OK
\end{cppcode}

尽管不知道具体的值,但该字符串处于一致状态。例如,\cppinline{s.size()}将返回字符数,可以遍历所有有效的索引:

\begin{cppcode}
foo(std::move(s)); // keeps s in a valid but unclear state

for (int i = 0; i < s.size(); ++i) {
	std::cout << s[i]; // OK
}
\end{cppcode}

对于用户定义类型,还应该确保移动的对象处于有效状态,有时需要声明或实现移动操作。“移动后的状态”这一章将对此进行详细讨论。

\subsection{重用移动的对象}

您可能想知道为什么已移动的对象仍然是有效的对象,并且没有(部分)销毁。原因是使用了移动语义,在这些应用中再次使用已移动的对象是有意义的。

例如,考虑从流中逐行读取字符串并将其移动到vector对象中的代码:

\begin{cppcode}
std::vector<std::string> allRows;
std::string row;
while (std::getline(myStream, row)) { // read next line into row
	allRows.push_back(std::move(row)); // and move it to somewhere
}
\end{cppcode}

每次将一行读入行后,使用\cppinline{std::move()}将\cppinline{row}的值移动到所有行的向量中。然后,s\cppinline{td::getline()}再次使用已移动的对象行来读入下一行。

第二个例子,考虑一个交换两个值的泛型函数:

\begin{cppcode}
template<typename T>
void swap(T& a, T& b)
{
	T tmp{std::move(a)};
	a = std::move(b); // assign new value to moved-from a
	b = std::move(tmp); // assign new value to moved-from b
}
\end{cppcode}

我们将\cppinline{a}的值移动到一个临时对象中,以便之后能够移动并赋\cppinline{b}的值。然后,被移动的对象\cppinline{b}接收到\cppinline{tmp}的值,这是\cppinline{a}之前的值。

例如,在排序算法中移动不同元素的值,使它们处于有序状态。给已移动的对象赋新值总是发生在那里。该算法甚至可以对这种移动对象的方式使用排序。

通常,已移动的对象是可以销毁的有效对象(析构函数不应该失败),重用以获得其他值,并支持类型的所有操作对象,而不需要知道具体值。“已移动的状态”这一章将对此进行详细讨论。

\subsection{将对象赋值给自己}

已移动的对象处于有效但未定义状态的规则,通常也适用于直接或间接自移动的对象。

例如,下面的语句之后,对象\cppinline{x}通常在不知道其值的情况下是有效的:

\begin{cppcode}
x = std::move(x); // afterwards x is valid but has an unclear value
\end{cppcode}

同样,C++标准库保证对用户定义类型的实例对象通常也提供这种保证,但有时必须来修复默认生成的移动状态。


