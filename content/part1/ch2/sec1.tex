\section{右值引用}

为了支持移动语义,C++引入了一种新的引用类型:\textit{右值引用}。我们讨论一下这是什么,以及如何使用。

\subsection{细节部分}

右值引用使用两个\&号声明。与普通引用一样,右值引用了一个存在的对象,该对象作为初始值传递。但根据语义,右值引用只能引用没有名称的临时对象,或使用\textit{std::move()}的对象:

\begin{cppcode}
std::string returnStringByValue(); // forward declaration
...
std::string s{"hello"};
...
std::string&& r1{s}; // ERROR
std::string&& r2{std::move(s)}; // OK
std::string&& r3{returnStringByValue()}; // OK, extends lifetime of return value
\end{cppcode}

右值引用来自这样一个事实:对象通常只能引用右值,这是类别,用于没有名称的临时对象和使用\textit{std::move()}的对象。

与成功初始化返回值引用一样,引用将返回值的生命周期延长到引用的生命周期结束(普通的\textit{const}左值引用已经具有此行为)。

用于初始化引用的语法无关紧要。使用等号、大括号或圆括号都可以:

\begin{cppcode}
std::string s{"hello"};
...
std::string&& r1 = std::move(s); // OK, rvalue reference to s
std::string&& r2{std::move(s)}; // OK, rvalue reference to s
std::string&& r3(std::move(s)); // OK, rvalue reference to s
\end{cppcode}

所有这些引用都具有这样的语义:“只要对象的状态有效,就可以窃取/修改引用的对象。”编译器不会检查这些语义,因此可以对该类型的任何非\textit{const}对象那样修改右值引用,也可能不做修改。如果对一个对象有一个右值引用,该对象可能会收到一个不同的值(可能是也可能不是一个默认构造对象的值),或者保留原始值。

移动语义允许我们使用不再需要的值进行优化。如果编译器自动检测到从生命周期结束的对象中获取值,将自动切换到移动语义:

\begin{itemize}
	\item 传递一个临时对象的值,该对象将在语句执行后自动撤销。
	\item 传递一个使用\textit{std::move()}的非\textit{const}对象。
\end{itemize}

\subsection{作为参数的右值引用}

将形参声明为右值引用时,具有的行为和语义:

\begin{itemize}
	\item 形参只能绑定到一个没有名称的临时对象,或者绑定到一个使用\textit{std::move()}的对象。
	\item 根据右值引用的语义:
	\begin{itemize}
		\item[-] 调用者不再对值感兴趣。因此,可以修改参数所引用的对象。
		\item[-] 但是,调用者可能仍然对使用对象感兴趣。因此,任何修改都应该使引用的对象保持有效状态。
	\end{itemize}
\end{itemize}

例如:

\begin{cppcode}
void foo(std::string&& rv); // takes only objects where we no longer need the value
...
std::string s{"hello"};
...
foo(s); // ERROR
foo(std::move(s)); // OK, value of s might change
foo(returnStringByValue()); // OK
\end{cppcode}

可以在通过\textit{std::move()}传递命名对象后使用,但通常不这样做。推荐的编程方式是不在\textit{std::move()}后面使用对象:

\begin{cppcode}
void foo(std::string&& rv); // takes only objects where we no longer need the value
...
std::string s{"hello"};
...
foo(std::move(s)); // OK, value of s might change
std::cout << s << '\n'; // OOPS, you don’t know which value is printed
foo(std::move(s)); // OOPS, you don’t know which value is passed
s = "hello again"; // OK, but rarely done
foo(std::move(s)); // OK, value of s might change
\end{cppcode}

对于标记为“OOPS”的两行,只要不对\textit{s}的当前值有期望,技术上是可以调用的。因此,打印是可以的。




