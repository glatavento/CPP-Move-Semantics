\section{按值传递}
当声明一个函数以值作为参数,会(自动)使用移动语义。

例如:

\begin{cppcode}
void foo(std::string str); // takes the object by value
...
std::string s{"hello"};
...
foo(s); // calls foo(), str becomes a copy of s
foo(std::move(s)); // calls foo(), s is moved to str
foo(returnStringByValue()); // calls foo(), return value is moved to str
\end{cppcode}

如果调用者发出信号表示不再需要传递的实参的值(通过\cppinline{std::move()}或传递一个没有名称的临时对象),形参\cppinline{str}将用从传递的实参中窃取值进行初始化。

这意味着,如果使用移动语义传递了一个临时对象或传递的参数标记为\cppinline{std::move()},那么按值调用会突然变得便利起来。就像按值返回本地对象一样,这个移动也可以优化掉。如果没有优化掉,那么这个调用现在肯定是高效的(如果有更高效的移动语义的话)。

注意以下差异:

\begin{cppcode}
void fooByVal(std::string str); // takes the object by value
void fooByRRef(std::string&& str); // takes the object by rvalue reference
...
std::string s1{"hello"}, s2{"hello"};
...
fooByVal(std::move(s1)); // s1 is moved
fooByRRef(std::move(s2)); // s2 might be moved
\end{cppcode}

这里,我们比较两个函数:一个以值作为字符串,另一个以字符串作为右值引用。在这两种情况下,都传递了带有\cppinline{std::move()}的字符串。

\begin{itemize}
	\item 按值获取字符串的函数将使用移动语义,因为新字符串是用传递的参数的值创建的。
	\item 通过右值引用获取字符串的函数可以使用移动语义,传递参数不会创建新字符串。是否窃取/修改传递的参数值取决于函数的实现。
\end{itemize}

因此:

\begin{itemize}
	\item 声明为支持移动语义的函数可能不使用移动语义。
	\item 通过值声明接受参数的函数将使用移动语义。
\end{itemize}

移动语义的效果并不能保证优化发生和优化效果。我们知道的是,传递的对象随后处于有效但未定义的状态。













