\section{std::move()}
如果有一个对象,当使用的时候,生存期没有结束,可以用 \cppinline{std::move()} 标记它,表示“在这里不再需要这个值。”\cppinline{std::move()} 不进行移动,只在使用表达式的上下文中设置了一个临时标记:

\begin{cppcode}
void foo1(const std::string& lr); // binds to the passed object without modifying it
void foo1(std::string&& rv); // binds to the passed object and might steal/modify the value
...
std::string s{"hello"};
...
foo1(s); // calls the first foo1(), s keeps its value
foo1(std::move(s)); // calls the second foo1(), s might lose its value
\end{cppcode}

带有 \cppinline{std::move()} 标记的对象仍然可以传递给接受普通 \cppinline{std::move()} 左值引用的函数:

\begin{cppcode}
void foo2(const std::string& lr); // binds to the passed object without modifying it
... // no other overload of foo2()
std::string s{"hello"};
...
foo2(s); // calls foo2(), s keeps its value
foo2(std::move(s)); // also calls foo2(), s keeps its value
\end{cppcode}

注意,用 \cppinline{std::move()} 标记的对象不能传递给非 \cppinline{std::move()} 左值引用:

\begin{cppcode}
void foo3(std::string&); // modifies the passed argument
...
std::string s{"hello"};
...
foo3(s); // OK, calls foo3()
foo3(std::move(s)); // ERROR: no matching foo3() declared
\end{cppcode}

注意,用 \cppinline{std::move()} 标记马上会销毁的对象是没有意义的。事实上,这甚至会对优化产生反效果。

\subsection{std::move()的头文件}

\cppinline{std::move()} 定义为C++标准库中的一个函数。因此,使用时需要包含头文件<utility>:

\begin{cppcode}
#include <utility> // for std::move()
\end{cppcode}

使用 \cppinline{std::move()} 的程序在编译时通常不包含这个头文件,因为几乎所有的头文件都包含了<utility>。但是,非标准头文件需要包含该头文件。因此,当使用 \cppinline{std::move()} 时,应该显式包含<utility>以使程序可移植。

\subsection{实现std::move()}

\cppinline{std::move()} 只不过是对右值引用的 \cppinline{static_cast}。可以手动调用 \cppinline{static_cast} 来达到相同的效果,如下所示:

\begin{cppcode}
foo(static_cast<decltype(obj)&&>(obj)); // same effect as foo(std::move(obj))
\end{cppcode}

因此,我们也可以这样写:

\begin{cppcode}
std::string s;
...
foo(static_cast<std::string&&>(s)); // same effect as foo(std::move(s))
\end{cppcode}

注意,\cppinline{static_cast} 所做的不仅仅是改变对象的类型。还允许将对象传递给右值引用(记住,通常不允许将具有名称的对象传递给右值引用)。我们将在关于价值类别的章节中详细讨论。





























