\section{noexcept在类中的声明}
可以看到,特别是必须实现一个移动构造函数时,应该用noexcept保证来声明。通常,遵循C++标准的规则,当所有基类和所有成员类型在移动赋值时都没有抛出异常时,应该声明为noexcept。

通常的模式如下:

\begin{cppcode}
class Base {
	...
};

class Drv : public Base {
	MemType member;
	...
	// move constructor:
	Drv(Drv&&) noexcept(std::is_nothrow_move_constructible_v<Base> &&
	std::is_nothrow_move_constructible_v<MemType>);
};
\end{cppcode}

这里,\cppinline{Drv}类的移动构造函数保证,如果基类\cppinline{Base}和成员类型\cppinline{MemType}提供了这个保证,则不会抛出异常。

移动赋值操作符可能使用相同的模式。但请注意,多态类型中应该删除移动赋值操作符,所以通常不需要在派生类中实现它们。

\subsection{检查抽象基类中的移动构造函数}

请注意,类型特征std::is_nothrow_move_constructible<>并不会总如预期。对于抽象基类,总是产生\cppinline{false},因为还会检查是否可以用移动构造函数创建该类型的对象,而这对于抽象类型是不可能的。

因此,“如果抽象基类保证不抛出异常,则继承类也应该保证不抛出异常”的声明不能使用标准类型特征来表述。通常,只需(必须)知道基类的移动构造函数是否可能抛出。

为了能够检查一个类的移动构造函数是否保证不会抛出,可以实现以下helper类型特征(使用C++20实现)。但请注意,必须为每个纯虚函数提供实现:

\filename{poly/isnothrowmovable.hpp}
\begin{cppcode}
// type trait to check whether a base class guarantees not to throw
// in the move constructor (even if the constructor is not callable)
#ifndef IS_NOTHROW_MOVABLE_HPP
#define IS_NOTHROW_MOVABLE_HPP

#include <type_traits>

template<typename Base>
struct Wrapper : Base {
	using Base::Base;
	// implement all possibly wrapped pure virtual functions:
	void print() const {}
	...
};

template<typename T>
static constexpr inline bool is_nothrow_movable_v
	= std::is_nothrow_move_constructible_v<Wrapper<T>>;

#endif // IS_NOTHROW_MOVABLE_HPP
\end{cppcode}

现在甚至可以检查抽象基类的移动构造函数是否为noexcept。下面的程序演示了标准和用户定义的类型特征的不同行为:

\filename{poly/isnothrowmovable.cpp}
\begin{cppcode}
#include "isnothrowmovable.hpp"
#include <iostream>

class Base {
	std::string id;
	...
public:
	virtual void print() const = 0; // pure virtual function (forces abstract base class)
	...
	virtual ~Base() = default;
	protected:
	// protected copy and move semantics (also forces abstract base class):
	Base(const Base&) = default;
	Base(Base&&) = default;
	// disable assignment operator (due to the problem of slicing):
	Base& operator= (Base&&) = delete;
	Base& operator= (const Base&) = delete;
};

int main()
{
	std::cout << std::boolalpha;
	std::cout << "std::is_nothrow_move_constructible_v<Base>: "
	<< std::is_nothrow_move_constructible_v<Base> << '\n';
	std::cout << "is_nothrow_movable_v<Base>: "
	<< is_nothrow_movable_v<Base> << '\n';
}
\end{cppcode}

该程序有以下输出:

\begin{outputcode}
std::is_nothrow_move_constructible<Base>: false
is_nothrow_movable<Base>: true
\end{outputcode}

因此,如果必须在抽象基类派生的类中实现移动构造函数,可以使用helper类型特征来声明移动构造函数。如下所示:

\begin{cppcode}
class Drv : public Base {
	MemType member;
	...
	// move constructor:
	Drv(Drv&&) noexcept(is_nothrow_movable_v<Base> &&
	is_nothrow_movable_v<MemType>);
};
\end{cppcode}

由于不需要实现所有纯虚函数,C++标准中会缺少编译器支持的类型特征。



























