\section{noexcept声明的细节}
正如所见,noexcept的引入是为了允许条件保证不抛出异常。通常,在编译时知道函数不能抛出异常可以改进代码和优化,因为不必处理由于抛出异常而进行的清理。注意,如果违反了noexcept保证,程序会调用std::terminate(),而后者通常会调用std::abort()来导致“异常的程序终止”(例如,core dump)。

\subsection{声明noexcept的函数的规则}

当声明noexcept条件时,有几个适用的规则:

\begin{itemize}
	\item noexcept条件必须是编译时表达式,该表达式的值可转换为bool类型。
	\item 不能重载不同条件的函数。
	\item 在类层次结构中,noexcept条件是接口的一部分。用不是noexcept的函数覆盖不是noexcept的基类函数是错误的。
\end{itemize}

例如:

\begin{cppcode}
class Base {
	public:
	...
	virtual void foo(int) noexcept;
	virtual void foo(int); // ERROR: overload on different noexcept clause only
	virtual void bar(int);
};

class Derived : public Base {
	public:
	...
	virtual void foo(int) override; // ERROR: override giving up the noexcept guarantee
	virtual void bar(int) noexcept; // OK (here we also guarantee not to throw)
};
\end{cppcode}

但是,对于非虚函数,派生类成员可以使用不同的noexcept声明隐藏基类成员:

\begin{cppcode}
class Base {
	public:
	...
	void foo(int) noexcept;
};

class Derived : public Base {
	public:
	...
	void foo(int); // OK, hiding instead of overriding
};
\end{cppcode}

条件在编译时计算后遵循相同的规则。例如,考虑以下类的层次结构:

\begin{cppcode}
class Base {
	public:
	virtual void func() noexcept(sizeof(int) < 8); // might throw if sizeof(int) >= 8
};

class Derived : public Base {
	public:
	void func() noexcept(sizeof(int) < 4) override; // might throw if sizeof(int) >= 4
};
\end{cppcode}

这种情况下,如果int的大小为4,将出现编译时错误,因为基类保证在调用\textit{func()}时不抛出,而派生类不再为\textit{func()}提供这种保证。当int的大小小于4时,两者都是noexcept,这挺不错。当int的大小至少为8时,两者都不是noexcept,这也没问题。因此,派生类只需要进一步限制异常保证。

\subsection{noexcept的特殊成员函数}

不能为特殊成员函数自动生成条件。

\subsubsection{noexcept的复制和移动特殊成员函数}

按照规则,当生成但未实现特殊成员函数时,将生成noexcept条件。这种情况下,如果对所有基类和非静态成员使用相应操作应该保证不抛出异常。

例如:

\filename{basics/specialnoexcept.cpp}
\begin{cppcode}
#include <iostream>
#include <type_traits>

class B
{
	std::string s;
};

int main()
{
	std::cout << std::boolalpha;
	std::cout << std::is_nothrow_default_constructible<B>::value << '\n';
	std::cout << std::is_nothrow_copy_constructible<B>::value << '\n';
	std::cout << std::is_nothrow_move_constructible<B>::value << '\n';
	std::cout << std::is_nothrow_copy_assignable<B>::value << '\n';
	std::cout << std::is_nothrow_move_assignable<B>::value << '\n';
}
\end{cppcode}

程序输出如下:

\begin{outputcode}
true
false
true
false
true
\end{outputcode}

生成的复制构造函数和复制赋值操作符可能会抛出异常,因为复制std::string可能会抛出异常。但生成的默认构造函数、移动构造函数和移动赋值操作符保证不会抛出异常,因为类std::string的默认构造函数、移动构造函数和移动赋值操作符保证不会抛出异常。

请注意,noexcept条件甚至在这些特殊成员函数使用\textit{=default}进行声明时也会生成。因此,如果像下面这样声明B类,效果是一样的:

\begin{cppcode}
class B
{
	std::string s;
public:
	B(const B&) = default; // noexcept condition automatically generated
	B(B&&) = default; // noexcept condition automatically generated
	B& operator= (const B&) = default; // noexcept condition automatically generated
	B& operator= (B&&) = default; // noexcept condition automatically generated
};
\end{cppcode}

当有一个默认的特殊成员函数时,可以显式地指定不同于生成的noexcept保证。例如:

\begin{cppcode}
class C
{
	...
public:
	C(const C&) noexcept = default; // guarantees not to throw (OK since C++20)
	C(C&&) noexcept(false) = default; // specifies that it might throw (OK since C++20)
	...
};
\end{cppcode}

C++20之前,如果生成的noexcept条件与指定的noexcept条件相矛盾,则删除已定义的函数。

\subsubsection{noexcept的析构函数}

按照规则,析构函数总是保证在默认情况下不会抛出异常。这既适用于生成的析构函数,也适用于实现的析构函数。

例如:

\begin{cppcode}
class B
{
	std::string s;
	public:
	...
	~B() { // automatically always declared as ~B() noexcept
		...
	}
};
\end{cppcode}

对于noexcept(false),可以在没有这个保证的情况下进行声明,但这通常没有意义,因为C++标准库保证析构函数从不抛出异常。


































