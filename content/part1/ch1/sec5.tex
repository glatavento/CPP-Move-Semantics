\section{总结}
\begin{itemize}
	\item 移动语义允许对对象的复制进行优化,它可以隐式使用(用于未命名的临时对象或局部返回值),也可以显式使用(通过 \cppinline{std::move()})。
	\item \cppinline{std::move()} 表示不再需要这个值,它将对象标记为可移动的。标记为 \cppinline{std::move()} 的对象不会(部分地)销毁(析构函数仍然会调用)。
	\item 通过使用非 \cppinline{const} 右值引用(例如 \cppinline{std::string&&})声明函数,可以定义一个接口,调用者在接口中从语义上声明不再需要传递的值。函数可以通过“窃取”这个信息来进行优化,或者对传递的参数做任何修改。通常,实现者还必须确保传递的参数在调用后处于有效状态。
	\item 移动的C++标准库的对象仍然是有效的对象,但其值为未定义。
	\item 拷贝语义用作移动语义的备选(如果拷贝语义支持的话)。如果没有采用右值引用的实现,则使用任何采用普通 \cppinline{const} 左值引用的实现(如:\cppinline{const std::string&})。即使对象被显式地标记为 \cppinline{std::move()},也会使用备选方式。
	\item 对 \cppinline{const} 对象调用 \cppinline{const} 通常没有效果。
	\item 如果按值(而不是按引用)返回,不要将返回值声明为 \cppinline{const}。
\end{itemize}

