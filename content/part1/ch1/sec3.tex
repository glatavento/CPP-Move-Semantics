\section{复制是一种应急方式}
可以通过移动语义,移动临时对象或用\cppinline{std::move()}标记的对象,可以通过“窃取”源值来优化值的复制(采用非\cppinline{const}右值引用)。但是,如果没有针对移动语义的优化版本,则使用复制。

例如,像vector这样的容器类缺少\cppinline{push_back()}的重载:

\begin{cppcode}
template<typename T>
class MyVector {
	public:
	...
	void push_back (const T& elem); // insert a copy of elem
	... // no other push_back() declared
};
\end{cppcode}

仍然可以传递一个临时对象或使用\cppinline{std::move()}:

\begin{cppcode}
MyVector<std::string> coll;
std::string s{"data"};
...
coll.push_back(std::move(s)); // OK, uses copy semantics
\end{cppcode}

对于临时对象或使用\cppinline{std::move()}的对象,首选是将形参声明为右值引用的函数。如果不存在这样的函数,则使用复制语义。这样,就可以确保调用者不需要知道是否存在优化。优化有可能没有,因为:


\begin{itemize}
	\item 函数/类是在移动语义支持之前实现,或者没有考虑支持移动语义
	\item 没有什么需要优化的(只有数字成员的类就是一个例子)
\end{itemize}

对于泛型代码,如果不再需要一个对象的值,可以用\cppinline{std::move()}。即使没有移动语义的支持,相应的代码也会编译通过。


出于同样的原因,甚至可以用\cppinline{std::move()}来标记基本数据类型的对象,如int(或指针)。仍然会使用复制的值(地址)方式:


\begin{cppcode}
std::vector<int> coll;
int x{42};
...
coll.push_back(std::move(x)); // OK, but copies x (std::move() has no effect)
\end{cppcode}


	