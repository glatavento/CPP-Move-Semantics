\section{\cppinline{const} 对象的移动语义}
最后,不能移动用 \cppinline{const} 声明的对象。因为任何优化实现都要求可以修改传递的实参,如果不允许修改,就不能窃取。

使用 \cppinline{push_back()} 的重载:

\begin{cppcode}
template<typename T>
class vector {
	public:
	...
	// insert a copy of elem:
	void push_back (const T& elem);
	// insert elem when the value of elem is no longer needed:
	void push_back (T&& elem);
	...
};
\end{cppcode}

对 \cppinline{const} 对象使用的函数是具有 \cppinline{const} 形参的 \cppinline{const} 重载:

\begin{cppcode}
std::vector<std::string> coll;
const std::string s{"data"};
...
coll.push_back(std::move(s)); // OK, calls push_back(const std::string&)
\end{cppcode}

\cppinline{const} 对象的 \cppinline{std::move()} 没起作用。

原则上,可以通过声明带有 \cppinline{const} 右值引用的函数进行重载,但在语义上没有意义。同样,\cppinline{const} 左值引用会作为处理这种情况的备选。

\subsection{\cppinline{const} 返回值}

\cppinline{const} 禁用移动语义对声明返回类型也有影响。\cppinline{const} 返回值不能移动。

因此,从C++11开始,用 \cppinline{const} 返回值就不再是好的方式了(正如过去的一些风格指南所推荐的那样)。例如:

\begin{cppcode}
const std::string getValue();

std::vector<std::string> coll;
...
coll.push_back(getValue()); // copies (because the return value is const)
\end{cppcode}

当按值返回时,不要将整个返回值声明为 \cppinline{const}。仅在声明部分返回类型时使用 \cppinline{const}(例如:返回的引用或指针所指向的对象):

\begin{cppcode}
const std::string getValue(); // BAD: disables move semantics for return values
const std::string& getRef(); // OK
const std::string* getPtr(); // OK
\end{cppcode}





