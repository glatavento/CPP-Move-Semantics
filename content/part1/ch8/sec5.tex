\section{当rvalue变成lvalue}
现在让了解一下将形参声明为rvalue引用的函数的实现:

\begin{cppcode}
void rvFunc(std::string&& str) {
	...
}
\end{cppcode}

只能传递rvalue:

\begin{cppcode}
std::string s{ ... };
rvFunc(s); // ERROR: passing an lvalue to an rvalue reference
rvFunc(std::move(s)); // OK, passing an xvalue
rvFunc(std::string{"hello"}); // OK, passing a prvalue
\end{cppcode}

然而,当在函数内部使用\cppinline{str}形参时,处理的是有名称的对象。这意味着使用\cppinline{str}作为lvalue。

不能直接递归地调用自己的函数:

\begin{cppcode}
void rvFunc(std::string&& str) {
	rvFunc(str); // ERROR: passing an lvalue to an rvalue reference
}
\end{cppcode}

必须再次用\cppinline{std::move()}标记\cppinline{str}:

\begin{cppcode}
void rvFunc(std::string&& str) {
	rvFunc(std::move(str)); // OK, passing an xvalue
}
\end{cppcode}

这是没有传递移动语义规则的规范。这是特性,而不是bug。如果传递了移动语义,就不能使用两次传递了移动语义的对象,因为第一次使用后,就会失去它的值。或者,需要临时禁用移动语义的特性。

如果将rvalue引用参数绑定到rvalue(prvalue或xvalue),该对象将作为lvalue,必须再次将其转换为rvalue,以便传递给rvalue引用。

现在,请记住\cppinline{std::move()}只不过是对rvalue引用的\cppinline{static_cast}。也就是说,可以在递归调用中编写如下程序:

\begin{cppcode}
void rvFunc(std::string&& str) {
	rvFunc(static_cast<std::string&&>(str)); // OK, passing an xvalue
}
\end{cppcode}

将对象\cppinline{str}转换为string类型。通过强制转换,改变值的类型。根据规则,通过对rvalue引用的强制转换,lvalue变成了xvalue,因此允许将对象传递给rvalue引用。

这并不是什么新鲜事:即使在C++11之前,声明为lvalue引用的形参在使用时也遵循lvalue规则。关键是声明中的引用指定了可以传递给函数的内容。对于函数内部的行为,与引用无关。

困惑吗?这就是在C++标准中定义移动语义和值类型的规则。是否有足够的了解,其实并不重要,编译器明白这些规则其实就足够了。

这里需要了解的是移动语义没有传递。如果传递一个带有移动语义的对象,必须再次用\cppinline{std::move()}标记,将其语义转发给另一个函数。











































