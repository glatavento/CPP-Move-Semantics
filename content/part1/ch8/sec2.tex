\section{值类别的特殊规则}
对于影响移动语义的函数和成员的值类型,有特殊的规则。

\subsection{函数的值类型}

C++标准中的特殊规则是,所有引用函数的表达式都是lvalue。

例如:

\begin{cppcode}
void f(int) {
}

void(&fref1)(int) = f; // fref1 is an lvalue
void(&&fref2)(int) = f; // fref2 is also an lvalue

auto& ar = std::move(f); // OK: ar is lvalue of type void(&)(int)
\end{cppcode}

与对象类型相反,可以将非\cppinline{const} lvalue引用绑定到标记为\cppinline{std::move()}的函数指针,因为标记为\cppinline{std::move()}的函数指针仍然是lvalue。

\subsection{数据成员的值类型}

如果使用对象的数据成员(例如,使用std::pair<>的第一个和第二个成员时),将使用特殊规则。

通常,数据成员的值类型如下:

\begin{itemize}
	\item lvalue的数据成员是lvalue。
	\item rvalue的引用和静态数据成员是lvalue。
	\item rvalue的普通数据成员是xvalue。
\end{itemize}

这些规则反映了引用或静态成员实际上不是对象的一部分。如果不再需要对象的值,这也适用于对象的普通数据成员。但是,引用或静态的成员的值可能被其他对象所使用。

例如:

\begin{cppcode}
std::pair<std::string, std::string&> foo(); // note: member second is reference

std::vector<std::string> coll;
...
coll.push_back(foo().first); // moves because first is an xvalue here
coll.push_back(foo().second); // copies because second is an lvalue here
\end{cppcode}

需要使用\cppinline{std::move()}来移动第二个成员:

\begin{cppcode}
coll.push_back(std::move(foo().second)); // moves
\end{cppcode}

如果有lvalue(一个有名字的对象),就有两种使用\cppinline{std::move}的方式来移动成员:

\begin{itemize}
	\item std::move(obj).member
	\item std::move(obj.member)
\end{itemize}

\cppinline{std::move()}的意思是“不再需要这个值”,所以不再需要对象的值,应该标记\cppinline{obj}。如果不再需要成员的值,应该标记\cppinline{member}。然而,实际情况会比较复杂。

\subsubsection{\cppinline{std::move()}用于普通数据成员}

如果成员既不是静态也不是引用\cppinline{,std::move()}能将成员转换为xvalue,以便能够使用移动语义。

考虑声明了以下内容:

\begin{cppcode}
std::vector<std::string> coll;
std::pair<std::string, std::string> sp;
\end{cppcode}

以下代码先将成员移动,然后再将成员移动到\cppinline{coll}中:

\begin{cppcode}
sp = ... ;
coll.push_back(std::move(sp.first)); // move string first into coll
coll.push_back(std::move(sp.second)); // move string second into coll
\end{cppcode}

但是,下面的代码具有相同的效果:

\begin{cppcode}
sp = ... ;
coll.push_back(std::move(sp).first); // move string first into coll
coll.push_back(std::move(sp).second); // move string second into coll
\end{cppcode}

看起来有点奇怪,\cppinline{std::move()}标记对象之后仍然使用obj。在本例中,知道对象的哪个部分可以移动,所以可以使用未移动的部分。因此,当必须实现移动构造函数时,我更喜欢用\cppinline{std::move()}标记成员。

\subsubsection{\cppinline{std::move()}用于引用或静态成员}

如果成员是引用或静态的,则使用不同的规则:rvalue的引用或静态成员是lvalue。同样,这条规则反映了,成员的值并不是对象的一部分。“不再需要对象的值”并不意味着“不再需要不属于对象的值(成员的值)”。

因此,如果有引用或静态成员,那么如何使用\cppinline{std::move()}是有区别的:

\begin{itemize}
	\item 对对象使用\cppinline{std::move()}不起作用:

	\begin{cppcode}
struct S {
	static std::string statString; // static member
	std::string& refString; // reference member
};
S obj;
...
coll.push_back(std::move(obj).statString); // copies statString
coll.push_back(std::move(obj).refString); // copies refString
	\end{cppcode}
	\item 对成员使用\cppinline{std::move(})具有的效果:

	\begin{cppcode}
struct S {
	static std::string statString;
	std::string& refString;
};
S obj;
...
coll.push_back(std::move(obj.statString); // moves statString
coll.push_back(std::move(obj.refString); // moves refString
	\end{cppcode}
	这样的举措是否有用是另一个问题。窃取静态成员或引用成员的值意味着修改所使用对象外部的值,这还能说得通,但也可能是意外和危险的。通常,类型应该更好地保护对这些成员的访问。

\end{itemize}

泛型代码中,可能不知道成员是静态的还是引用的。因此,使用\cppinline{std::move()}来标记对象是不那么危险的,就是看起来奇怪:

\begin{cppcode}
coll.push_back(std::move(obj).mem1); // move value, copy reference/static
coll.push_back(std::move(obj).mem2); // move value, copy reference/static
\end{cppcode}

稍后将介绍的\cppinline{std::forward<>()}可以用来完美地转发对象的成员。参见basics/members.cpp获取完整的示例。
























