\section{类中使用移动语义}
任何复制构造函数、复制赋值或析构函数的声明都禁用了对移动语义的自动支持。这也适用于多态基类。然而,还有一些其他方面需要考虑。

\subsection{实现多态基类}

多态基类通常引入虚成员函数,可以调用派生类的所有对象。例如:

\begin{cppcode}
class GeoObj {
	public:
	virtual void draw() const = 0; // pure virtual function (introducing the API)
	...
	virtual ~GeoObj() = default; // let delete call the right destructor
	... // other special member functions due to the problem of slicing
};
\end{cppcode}

这个基类中,禁用移动语义,如果移动\cppinline{GeoObj}对象,基类中声明的成员不会自动支持移动语义。如果有受保护的复制构造函数和一个删除的赋值操作符也适用,通常在多态基类中应该有这样的操作符,避免切片问题。

只要基类不引入成员,不支持移动语义就没有效果。但是,如果这个基类中有一个开销很大的成员,就已经禁用了对移动语义的支持。例如:

\begin{cppcode}
class GeoObj {
protected:
	std::string name; // name of the geometric object
public:
	...
	virtual void draw() const = 0; // pure virtual function (introducing the API)
	...
	virtual ~GeoObj() = default; // disables move semantics for name
	... // other special member functions due to the problem of slicing
};
\end{cppcode}

要再次启用移动语义,可以显式声明移动操作为默认值。但正如我们刚刚了解到的,这禁用了复制特殊成员函数。因此,如果想要使用这些函数,就必须显式地提供。

\subsubsection{处理切片}

但是,有切片的问题。考虑以下代码,使用基类\cppinline{GeoObj}的引用作为派生类\cppinline{Circle}的对象:

\begin{cppcode}
Circle c1{ ... }, c2{ ... };

GeoObj& geoRef{c1};
geoRef = c2; // OOPS: uses GeoObj::operator=() and assigns no Circle members
\end{cppcode}

为\cppinline{GeoObj}调用赋值操作符,而且该操作符不是虚操作符,所以编译器调用\cppinline{GeoObj::operator=()},不处理任何派生类的成员。即使用\cppinline{virtual}声明赋值操作符也没有帮助,因为派生类的操作符不会覆盖基类的赋值操作符(第二个操作数的形参类型不同)。

为了避免这个问题,应该禁用在多态类层次结构中使用赋值操作符。此外,如果不是抽象类,还应该避免使用公共复制构造函数来禁用到基类的隐式类型转换。因此,具有移动语义(和成员)的多态基类应该如下声明:

\begin{cppcode}
class GeoObj {
protected:
	std::string name; // name of the geometric object
	GeoObj(std::string n)
	: name{std::move(n)} {
	}
public:
	virtual void draw() const = 0; // pure virtual function (introducing the API)
	...
	virtual ~GeoObj() = default; // would disable move semantics for name
protected:
	// enable copy and move semantics (callable only for derived classes):
	GeoObj(const GeoObj&) = default;
	GeoObj(GeoObj&&) = default;
	// disable assignment operator (due to the problem of slicing):
	GeoObj& operator= (GeoObj&&) = delete;
	GeoObj& operator= (const GeoObj&) = delete;
};
\end{cppcode}

参见\fninline{poly/geoobj.hpp}获取完整的头文件。

\subsection{实现派生类的多态}

派生类的多态看起来如下所示(参见poly/polygon.hpp的完整头文件):

\begin{cppcode}
class Polygon : public GeoObj {
protected:
	std::vector<Coord> points;
public:
	Polygon(std::string s, std::initializer_list<Coord> = {}); // constructor
	virtual void draw() const override; // implementation of draw()
};
\end{cppcode}

通常,多态派生类中不需要声明特殊的成员函数。特别是,不需要再次声明虚析构函数(除非必须实现)。再次声明析构函数(无论是否是虚函数)将禁用派生类成员(这里是vector)对移动语义的支持:

\begin{cppcode}
class Polygon : public GeoObj {
protected:
	std::vector<Coord> points;
public:
	Polygon(std::string s, std::initializer_list<Coord> = {}); // constructor
	...
	virtual ~Polygon() = default; // OOPS: don’t do that because it disables move semantics
};
\end{cppcode}

在不声明析构函数的情况下,移动语义适用于\cppinline{Polygon}成员、\cppinline{name}和\cppinline{points}。

\filename{poly/polygon.cpp}
\begin{cppcode}
#include "geoobj.hpp"
#include "polygon.hpp"

int main()
{
	Polygon p0{"Poly1", {Coord{1,1}, Coord{1,9}, Coord{9,9}, Coord{9,1}}};
	Polygon p1{p0}; // copy
	Polygon p2{std::move(p0)}; // move

	p0.draw();
	p1.draw();
	p2.draw();
}
\end{cppcode}

这个程序有以下输出:

\begin{outputcode}
polygon '' over
polygon 'Poly1' over (1,1) (1,9) (9,9) (9,1)
polygon 'Poly1' over (1,1) (1,9) (9,9) (9,1)
\end{outputcode}

对于这两个成员,\cppinline{name}和\cppinline{points},值都从\cppinline{p0}移动到\cppinline{p2}。

注意,如果必须在\cppinline{Polygon}类中实现移动构造函数,需要特别注意提供正确的条件。



























































