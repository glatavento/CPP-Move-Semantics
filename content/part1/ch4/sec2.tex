\section{避免不必要的\textit{std::move()}}
如果支持的话,按值返回本地对象会自动使用移动语义。然而,为了安全起见,开发者可能会尝试使用显式\textit{std::move()}强制执行:

\begin{cppcode}
std::string foo()
{
	std::string s;
	...
	return std::move(s); // BAD: don’t do this
}
\end{cppcode}

请记住,\textit{std::move()}只是对右值引用的\textit{static_cast}。因此,\textit{std::move(s)}会产生std::string\&\&类型的表达式。然而,这不再匹配返回类型,因此禁用了返回值优化,这通常允许返回对象作为返回值使用。对于没有实现移动语义的类型,这甚至会强制复制返回值,而不是仅仅使用返回的对象作为返回值。

因此,如果按值返回局部对象时,不要使用\textit{std::move()}:

\begin{cppcode}
std::string foo()
{
	std::string s;
	...
	return s; // best performance (return value optimization or move)
}
\end{cppcode}

临时对象时使用\textit{std::move()}是多余的。对于按值返回对象的c\textit{reateString()}函数,应该只使用返回值:

\begin{cppcode}
std::string s{createString()}; // OK
\end{cppcode}

而不是用\textit{std::move()}再次标记:

\begin{cppcode}
std::string s{std::move(createString())}; // BAD: don’t do this
\end{cppcode}

编译器可能(可以选择)对任何适得其反或不必要的\textit{std::move()}使用发出警告。例如,gcc有选项-Wpessimizing-move(通过-Wall启用)和-Wredundancy-move(通过-Wextra启用)。

在某些应用程序中,返回语句中的\textit{std::move()}是合适的。一个是移出成员的值,另一个是返回带有移动语义的参数。
















