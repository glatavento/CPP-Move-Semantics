\section{字符串的移动语义}
字符串是可以分配内存来保存其值的对象。因此,可以对其使用移动语义。

书中已经有了几个关于字符串如何支持移动语义的例子:

\begin{itemize}
	\item 将字符串移动到某个vector中
	\item 为字符串实现移动构造函数
	\item 成员初始化字符串
	\item 循环中读取字符串,并在\textit{move()}之后使用它们
\end{itemize}

本节中,将更详细地介绍移动语义对字符串的影响。

\subsection{字符串分配和容量}

字符串的容量(当前可用于该值的内存)通常不会减少。只有移动操作、swap()或shrink_to_fit()可以减小容量。

考虑下面例子:

\filename{lib/stringmoveassign.cpp}
\begin{cppcode}
#include <iostream>
#include <string>

int main()
{
	std::string s0;
	std::string s1{"short"};
	std::string s2{"a string with an extraordinarily long value"};
	std::cout << "- s0 capa: " << s0.capacity() << " ('" << s0 << "')\n";
	std::cout << " s1 capa: " << s1.capacity() << " ('" << s1 << "')\n";
	std::cout << " s2 capa: " << s2.capacity() << " ('" << s2 << "')\n";

	std::string s3{std::move(s1)};
	std::string s4{std::move(s2)};
	std::cout << "- s1 capa: " << s1.capacity() << " ('" << s1 << "')\n";
	std::cout << " s2 capa: " << s2.capacity() << " ('" << s2 << "')\n";
	std::cout << " s3 capa: " << s3.capacity() << " ('" << s3 << "')\n";
	std::cout << " s4 capa: " << s4.capacity() << " ('" << s4 << "')\n";

	std::string s5{"quite a reasonable value"};
	std::cout << "- s4 capa: " << s4.capacity() << " ('" << s4 << "')\n";
	std::cout << " s5 capa: " << s5.capacity() << " ('" << s5 << "')\n";

	s4 = std::move(s5);
	std::cout << "- s4 capa: " << s4.capacity() << " ('" << s4 << "')\n";
	std::cout << " s5 capa: " << s5.capacity() << " ('" << s5 << "')\n";
}
\end{cppcode}

代码中,由于小字符串优化(SSO),即使是空字符串也有容纳某些字符的能力,这通常为字符串本身的值保留15或22个字节。除了SSO的大小之外,字符串在堆上分配内存,堆至少有存储值所需的大小。因此:

\begin{cppcode}
std::string s0;
std::string s1{"short"};
std::string s2{"a string with an extraordinarily long value"};
\end{cppcode}

可能会得到如下输出:

\begin{itemize}
	\item 平台 A:
	\begin{outputcode}
- s0 capa: 15 ('')
s1 capa: 15 ('short')
s2 capa: 43 ('a string with an extraordinarily long value')
	\end{outputcode}
	\item 平台 B:
	\begin{outputcode}
- s0 capa: 22 ('')
s1 capa: 22 ('short')
s2 capa: 47 ('a string with an extraordinarily long value')
	\end{outputcode}
\end{itemize}

这里,两个平台都支持SSO(最多15或22个字符),当只需要43个字符的内存时,第二个平台为47个字符分配内存。

所有平台上,已移动字符串通常为空,即使值没有存储在外部分配的内存中(因此我们必须复制所有字符)。

\begin{cppcode}
std::string s3{std::move(s1)};
std::string s4{std::move(s2)};
\end{cppcode}

会得到这样的结果:

\begin{itemize}
	\item 平台 A:
	\begin{outputcode}
- s1 capa: 15 ('')
s2 capa: 15 ('')
s3 capa: 15 ('short')
s4 capa: 43 ('a string with an extraordinarily long value')
	\end{outputcode}
	\item 平台 B:
	\begin{outputcode}
- s1 capa: 22 ('')
s2 capa: 22 ('')
s3 capa: 22 ('short')
s4 capa: 47 ('a string with an extraordinarily long value')
	\end{outputcode}
\end{itemize}

注意,这里不能保证\textit{s1}变成空的。C++标准库只保证从字符串中移动的字符串处于有效但未指定的状态,这意味着\textit{s1}的值仍然可以是“short”,或是其他值。

移动时分配不同的字符串值可能会缩小容量。示例程序的最后两个步骤基本上执行:

\begin{cppcode}
std::string s4{"a string with an extraordinarily long value"};
std::string s5{"quite a reasonable value"};
s4 = std::move(s5);
\end{cppcode}

在实践中会有以下输出:

\begin{itemize}
	\item 内存交换:
	\begin{outputcode}
- s4 capa: 43 ('a string with an extraordinarily long value')
s5 capa: 24 ('quite a reasonable value')
- s4 capa: 24 ('quite a reasonable value')
s5 capa: 43 ('')
	\end{outputcode}
	\item 移动内存(释放旧内存后):
	\begin{outputcode}
- s4 capa: 47 ('a string with an extraordinarily long value')
s5 capa: 31 ('quite a reasonable value')
- s4 capa: 31 ('quite a reasonable value')
s5 capa: 22 ('')
	\end{outputcode}
\end{itemize}

\textit{s4}的容量通常会缩小,有时缩小到目标的容量,有时缩小到空字符串的最小容量。然而,两者都不能完全保证。
