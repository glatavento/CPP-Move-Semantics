\section{容器的移动语义}
容器通常是必须分配内存来保存其元素的对象。因此,可以从使用移动语义中获益。但有例外:std::array<>不在堆上分配内存,这意味着对std::array<>应该使用特殊规则。

书中已经有了几个关于容器如何支持移动语义的例子。最初的例子中,了解了支持的移动语义:

\begin{itemize}
	\item 通过重载\cppinline{push_back()},C++标准支持移动语义来插入新元素。
	\item 通过提供移动构造函数和移动赋值操作符,降低复制临时对象(比如返回值)的成本。
\end{itemize}

执行以下操作时,所有容器都支持移动语义:

\begin{itemize}
	\item 复制容器
	\item 赋值容器
	\item 插入容器
\end{itemize}

然而,还有更多。

\subsection{基本移动支持容器整体}

所有容器都定义了一个移动构造函数和移动赋值操作符,以支持未命名临时对象和\cppinline{std::move()}标记对象的移动语义。

例如,std::list<>的声明如下:

\begin{cppcode}
template<typename T, typename Allocator = allocator<T>>
class list {
	public:
	...
	list(const list&); // copy constructor
	list(list&&); // move constructor
	list& operator=(const list&); // copy assignment
	list& operator=(list&&) noexcept( ... ); // move assignment
	...
};
\end{cppcode}

这使得按值返回/传递容器并赋值的成本变低。例如:

\begin{cppcode}
std::list<std::string> createAndInsert()
{
	std::list<std::string> coll;
	...
	return coll; // move constructor if not optimized away
}
std::list<std::string> v;
...
v = createAndInsert(); // move assignment
\end{cppcode}

但请注意,这里有附加的要求和保证,这适用于除std::array<>之外的所有容器的移动构造函数和移动赋值操作符。这些要求和保证意味着已移动的容器移动是空的。

\subsubsection{容器移动构造函数的保证}

对于移动构造函数:

\begin{cppcode}
ContainerType cont1{ ... };
ContainerType cont2{std::move(cont1)}; // move the container
\end{cppcode}

C++标准规定了常量复杂度,移动的持续时间不取决于元素的数量。

有了这种保证,实现者没有其他选择,只能从源对象\cppinline{cont1}整体窃取元素的内存到目标对象\cppinline{cont2},使源对象\cppinline{cont1}处于初始/空状态。

可能会认为移动构造函数也可以在源对象中创建新值,但这没有多大意义,因为这只会使操作变慢。

对于vector,甚至间接禁止从已移动对象中获取值,因为std::vector<>的移动构造函数不会抛出异常:

\begin{cppcode}
template<typename T, typename Allocator = allocator<T>>
class vector {
	public:
	...
	vector(const vector&); // copy constructor
	vector(vector&&) noexcept; // move constructor
	...
};
\end{cppcode}

总之,在移动构造函数中使用容器作为源时,有以下保证:

\begin{itemize}
	\item 对于vector,本质上要求已移动的容器为空。
	\item 对于其他容器(除了std::array<>),不是严格要求为空,但实现为其他也没什么意义。
\end{itemize}

\subsubsection{容器移动赋值操作符的保证}

对于移动赋值操作符:

\begin{cppcode}
ContainerType cont1{ ... }, cont2{ ... };
cont2 = std::move(cont1); // move assign the container
\end{cppcode}

C++标准保证此操作会覆盖或销毁目标对象\cppinline{cont2}的每个元素。这保证了目标容器\cppinline{dest2}的元素在条目时拥有的所有资源都会进行释放。因此,只有两种方法来实现移动赋值:

\begin{itemize}
	\item 销毁旧的元素,并将源的全部内容移动到目标(即,将指向内存的指针从源移动到目标)。
	\item 一个元素一个元素地从源\cppinline{cont1}移动到目标\cppinline{cont2},并销毁目标中未覆盖的所有剩余元素。
\end{itemize}

这两种方法的复制度都线性的,在这个定义中,不允许仅交换源和目标的内容。

然而,自C++17以来,所有容器都保证在内存可互换时不会抛出异常。例如:

\begin{cppcode}
template<typename T, typename Allocator = allocator<T>>
class list {
public:
	...
	list& operator=(list&&)
	noexcept(allocator_traits<Allocator>::is_always_equal::value);
	...
};
\end{cppcode}

noexcept对赋值操作符的保证,排除了将移动赋值实现为逐个元素移动的第二种方法。移动操作可能会抛出异常,只有销毁旧元素的实现才不会抛出。因此,当内存可以互换时,必须使用第一种方法来实现移动赋值操作符。

总之,移动赋值中使用容器作为源,实际上有以下保证:

\begin{itemize}
	\item 如果内存是可互换的(使用默认标准分配器时尤其如此),则基本上要求从移动的容器为空。这适用于除std::array<>之外的所有容器。
	\item 否则,移动的容器处于有效但未定义的状态。
\end{itemize}

但请注意,在给自己移动赋值之后,容器总处于未定义但有效的状态。

\subsection{Insert和Emplace函数}

所有容器都支持将新元素移动到容器中。

\subsubsection{Insert函数}

例如,vector通过\cppinline{push_back()}的两种不同实现来支持移动语义:

\begin{cppcode}
template<typename T, typename Allocator = allocator<T>>
class vector {
public:
	...

	// insert a copy of elem:
	void push_back (const T& elem);

	// insert elem when the value of elem is no longer needed:
	void push_back (T&& elem);
	...
};
\end{cppcode}

rvalue的\cppinline{push_back()}函数用\cppinline{std::move()}传递传递的元素,这样就调用了元素类型的移动构造函数,而不是复制构造函数。

同样,所有容器都有相应的重载。例如:

\begin{cppcode}
template<typename Key, typename T, typename Compare = less<Key>,
typename Allocator = allocator<pair<const Key, T>>>
class map {
	public:
	...
	pair<iterator, bool> insert(const value_type& x);
	pair<iterator, bool> insert(value_type&& x);
	...
};
\end{cppcode}

\subsubsection{Emplace函数}

从C++11开始,容器也提供了Emplace函数(比如为向量提供了\cppinline{emplace_back()})。可以传递多个参数来直接在容器中初始化新元素,而不是传递单个元素类型参数(或可转换为元素类型)。这样就可以保存副本或移动。

注意,即使这样,容器也可以通过为构造函数的初始参数支持移动语义,并从移动语义中获益。

像\cppinline{emplace_back()}这样的函数可以使用完美转发来避免创建所传递参数的副本。例如,对于std::vector<>,emplace_back()成员函数的定义如下:

\begin{cppcode}
template<typename T, typename Allocator = allocator<T>>
class vector {
	public:
	...
	// insert a new element with perfectly forwarded arguments:
	template<typename... Args>
	constexpr T& emplace_back(Args&&... args) {
		...
		// call the constructor with the perfectly forwarded arguments:
		place_element_in_memory(T(std::forward<Args>(args)...));
		...
	}
	...
};
\end{cppcode}

在vector内部使用完全转发的参数初始化新元素。

\subsection{std::array<>的移动语义}

array<>是唯一没有在堆上分配内存的容器。实际上,是通过带有数组成员的模板化C数据结构实现的:

\begin{cppcode}
template<typename T, size_t N>
struct array {
	T elems[N];
	...
};
\end{cppcode}

因此,不能以移动指针到内部内存的方式来实现移动操作。

因此,std::array<>有两个保证:

\begin{itemize}
	\item 移动构造函数具有线性复杂度,必须逐个元素地移动。
	\item 移动赋值操作符可能总是抛出异常,因为它必须逐个元素地移动赋值。
\end{itemize}

因此,复制或移动一个数值数组没有区别:

\begin{cppcode}
std::array<double, 1000> arr;
...
auto arr2{arr}; // copies all double elements/values
auto arr3{std::move(arr)}; // still copies all double elements/values
\end{cppcode}

对于所有其他容器,后者将只移动指向新对象的内部指针,这是一个成本非常低的操作。

但是,如果移动元素比复制元素成本还要低,那么移动仍然比复制要好。例如:

\begin{cppcode}
std::array<std::string, 1000> arr;
...
auto arr2{arr}; // copies string by string
auto arr3{std::move(arr)}; // moves string by string
\end{cppcode}

如果字符串分配堆内存(即,如果使用小字符串优化(SSO),则有一个显著的大小),那么移动字符串数组通常会更快。

可以通过lib/contmove.cpp看到这一点,它检查复制和移动不同元素类型(双精度、小字符串和大字符串)的数组和vector之间的区别。请注意,在不同平台上,因为生成的代码具有不同的优化,复制和移动双精度数组或小字符串之间可能存在微小的性能差异。





















