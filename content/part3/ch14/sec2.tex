\section{移除性算法}
根据设计,C++算法使用迭代器来处理容器和范围的元素。然而,就像指针操作数组一样,迭代器只能读取和写入值,不能插入或删除元素。因此,“删除”算法并不是真正删除元素,只是将所有未移除元素的值移动到已处理范围的前面。

例如,给定以下整数序列:

\begin{outputcode}
1 2 3 4 5 4 3 2 1
\end{outputcode}

调用\cppinline{std::remove()}算法来删除值为2的所有元素,修改后的序列:

\begin{outputcode}
1 3 4 5 4 3 1 2 1
\end{outputcode}

所有不是2的元素都移到前面,作为新的结束(“last”元素后面的位置),返回2的位置。

可能的话,这些算法会移动,会以已移状态保留元素。本例中,如果元素是字符串,那么最后2个元素将保持不变,但最后1个元素将向前移动到2之前,以便最后一个元素处于已移动状态。

因此,这些算法也可以将元素保留为已移动状态。事实上,以下算法可以创建已移动状态:

\begin{itemize}
	\item \cppinline{std::remove()}和\cppinline{std::remove_if()}
	\item \cppinline{std::unique()}
\end{itemize}

看一个完整的例子,可以看到元素是否为已移动状态:

\filename{lib/email.hpp}
\begin{cppcode}
#include <iostream>
#include <cassert>
#include <string>

// class for email addresses
// - asserts that each email address has a @
// - except when in a moved-from state
class Email {
	private:
	std::string value; // email address
	bool movedFrom{false}; // special moved-from state
	public:
	Email(const std::string& val)
	: value{val} {
		assert(value.find('@') != std::string::npos);
	}
	Email(const char* val) // enable implicit conversions for string literals
	: Email{std::string(val)} {
	}
	std::string getValue() const {
		assert(!movedFrom); // or throw
		return value;
	}
	...
	// implement move operations to signal a moved-from state:
	Email(Email&& e) noexcept
	: value{std::move(e.value)}, movedFrom{e.movedFrom} {
		e.movedFrom = true;
	}
	Email& operator=(Email&& e) noexcept {
		value = std::move(e.value);
		movedFrom = e.movedFrom;
		e.movedFrom = true;
		return *this;
	}
	// enable copying:
	Email(const Email&) = default;
	Email& operator=(const Email&) = default;

	// print out the current state (even if it is a moved-from state):
	friend std::ostream& operator<< (std::ostream& strm, const Email& e) {
		return strm << (e.movedFrom ? "MOVED-FROM" : e.value);
	}
};
\end{cppcode}

通过实现移动构造函数和移动赋值操作符,设置了一个成员\cppinline{movedFrom},并在输出操作符中求值。

现在让我们将该类中的一些元素放入vector中,并使用算法\cppinline{std::remove_if()}删除一些元素。本例中,我们删除所有以".de"结尾的电子邮件地址:

\filename{lib/removeif.cpp}
\begin{cppcode}
#include "email.hpp"
#include <iostream>
#include <string>
#include <vector>
#include <algorithm>
int main()
{
	std::vector<Email> coll{ "tomdomain.de", "jillcompany.com",
		"sarahdomain.de", "hanacompany.com" };

	// remove all email addresses ending with ".de":
	auto newEnd = std::remove_if(coll.begin(), coll.end(),
	[] (const Email& e) {
		auto&& val = e.getValue();
		return val.size() > 2 &&
		val.substr(val.size()-3) == ".de";
	});

	// print elements up to the new end:
	std::cout << "remaining elements:\n";
	for (auto pos = coll.begin(); pos != newEnd; ++pos) {
		std::cout << " '" << *pos << "'\n";
	}

	// print all elements in the container:
	std::cout << "all elements:\n";
	for (const auto& elem : coll) {
		std::cout << " '" << elem << "'\n";
	}
}
\end{cppcode}

程序输出如下:

\begin{outputcode}
remaining elements:
"jill@company.com"
"hana@company.com"
all elements:
"jill@company.com"
"hana@company.com"
"sarah@domain.de"
"MOVED-FROM"
\end{outputcode}

“删除”以“.de”结尾的元素之后,新的结束位置是第二个元素后面的位置。但在容器中,第三个元素没有移动,第四个元素处于已移动状态,因为它移动赋值给了第二个元素(之前移动到第一个元素)。

使用这些已移动对象时要小心,原因在关于已移动状态的章节中已经讨论过了。

\subsection{移动算法中的迭代器}

算法中使用移动迭代器,通常只有在算法保证每个元素只使用一次时才有意义。因此,该算法应该:

\begin{itemize}
	\item 要求源的输入迭代器类别和目标的输出迭代器类别
	\item 或者保证每个元素只使用一次(例如,为\cppinline{std::for_each()}算法指定)
\end{itemize}

对于具有可调用对象的算法,允许规范提供详细的功能,元素通过\cppinline{std::move()}传递给可调用对象。在可调用对象内部,可以决定如何处理:

\begin{itemize}
	\item 按值接受参数,总是移动/窃取值或资源
	\item 通过右值/通用引用参数来决定移动/窃取哪个值/资源
\end{itemize}

例如:

\filename{lib/foreachmove.cpp}
\begin{cppcode}
#include <iostream>
#include <string>
#include <vector>
#include <algorithm>
template<typename T>
void print(const std::string& name, const T& coll)
{
	std::cout << name << " (" << coll.size() << " elems): ";
	for (const auto& elem : coll) {
		std::cout << " \"" << elem << "\"";
	}
	std::cout << "\n";
}
void process(std::string s) // gets moved value from rvalues
{
	std::cout << "- process(" << s << ")\n";
	...
}
int main()
{
	std::vector<std::string> coll{"don't", "vote", "for", "liars"};
	print("coll", coll);
	// move away only the elements processed:
	std::for_each(std::make_move_iterator(coll.begin()),
	std::make_move_iterator(coll.end()),
	[] (auto&& elem) {
		if (elem.size() != 4) {
			process(std::move(elem));
		}
	});
	print("coll", coll);
}
\end{cppcode}

代码中,将所有大小为4的元素移动到helper函数\cppinline{process()}中:

\begin{cppcode}
// move away only the elements processed:
std::for_each(std::make_move_iterator(coll.begin()),
std::make_move_iterator(coll.end()),
[] (auto&& elem) {
	if (elem.size() != 4) {
		process(std::move(elem));
	}
});
\end{cppcode}

为此,将标记的元素(移动迭代器标记为\cppinline{std::move()})通过通用引用传递给\cppinline{process()},并将其与\cppinline{std::move()}一起传递给\cppinline{process()}。因为\cppinline{process()}按值接受参数,所以值实际上已经移走。

因此,所有大小不为4的元素都从容器\cppinline{coll}中的对象中移出。该程序的输出如下(?表示未知值):

\begin{outputcode}
coll (4 elems): "don't" "vote" "for" "liars"
- process(don't)
- process(for)
- process(liars)
coll (4 elems): "?" "vote" "?" "?"
\end{outputcode}

最后一行是这样的:

\begin{outputcode}
coll (4 elems): "" "vote" "" ""
\end{outputcode}

这里使用了辅助函数\cppinline{std::make_move_iterator()},这样在声明迭代器时就不必指定元素类型了。从C++17开始,类模板实参推导(CTAD)允许直接声明类型\cppinline{std::move_iterator},而不需要指定元素类型:

\begin{cppcode}
std::for_each(std::move_iterator{coll.begin()},
std::move_iterator{coll.end()},
[] (auto&& elem) {
	if (elem.size() != 4) {
		process(std::move(elem));
	}
});
\end{cppcode}

\subsection{在构造函数和成员函数中移动迭代器}

也可以在只读取一次元素的算法中使用移动迭代器,场景可能是将源容器的元素移动到另一个容器(相同或不同类型的)。例如:

\filename{lib/moveitor.cpp}
\begin{cppcode}
#include <iostream>
#include <string>
#include <list>
#include <vector>

template<typename T>
void print(const std::string& name, const T& coll)
{
	std::cout << name << " (" << coll.size() << " elems): ";
	for (const auto& elem : coll) {
		std::cout << " \"" << elem << "\"";
	}
	std::cout << "\n";
}

int main()
{
	std::list<std::string> src{"don't", "vote", "for", "liars"};
	// move all elements from the list to the vector:
	std::vector<std::string> vec{std::make_move_iterator(src.begin()),
		std::make_move_iterator(src.end())};
	print("src", src);
	print("vec", vec);
}
\end{cppcode}

该程序有以下输出:(?表示未知值):

\begin{outputcode}
src (4 elems): "?" "?" "?" "?"
vec (4 elems): "don't" "vote" "for" "liars"
\end{outputcode}

请再次注意,源容器中的元素数量没有改变。所有元素会移动到初始化的新容器中。因此,源范围中的元素随后处于已移动状态,所以不知道它们的值。













