
\section*{本书概述}

完整的介绍C++ 移动语义。

C++11添加的移动语义已经成为现代C++的标志,也使语言变得复杂,即使经验丰富的开发者仍需要仔细处理。所以,一些编程书籍甚至不推荐对非常简单的类使用移动语义。因此,详细的解释C++移动语义就变得刻不容缓。

本书会从基本原理开始来介绍移动语义,并解释移动语义的细节,使每个开发者都可以正确地使用移动语义。

你将学习到:

\begin{itemize}
    \item 移动语义的起因和术语
    \item 如何隐式地获益于移动语义
    \item 如何显式地获益于移动语义
    \item 会遇到的问题,以及如何处理它们
    \item 所有的结果都取决于你的编程风格
\end{itemize}

重点在于所描述的特性,需要在实践中进行应用。示例和背景信息,有助于理解和改进简单类,甚至泛型库和框架的代码。

“我以为我理解了移动的语义,其实我真的不懂!我从你的书中学到了很多东西。”	-- Jonathan Boccara

“这是我非常需要的书。” -- Rob Bernstein

“有时候我觉得我对纠缠和量子隐形传态的理解,要比我对一些奇怪的C++ 移动语义的理解要好。套用Feynman的话:如果你认为你理解了C++的移动语义,那你就不理解C++的移动语义。赶快阅读这本书吧。”	-- Victor Ciura


\section*{作者简介}
Nicolai Josuttis (http://www.josuttis.com)在编程界很有名,其发言和著作都很有权威,还是世界范围内畅销书的(共同)作者:

\begin{itemize}
    \item 《The C++ Standard Library》
    \item 《C++ Templates》
    \item 《C++ Move Semantics》
    \item 《C++17》
    \item 《SOA in Practice》
\end{itemize}

同时也是一位富有创新精神的演讲者,曾在各种会议和活动中发言。还是独立的讲师,并且在C++标准化方面有20多年的经验。

\section*{本书相关}
\begin{itemize}
    \item github翻译地址:\href{https://github.com/xiaoweiChen/CPP-Move-Semantics}{https://github.com/xiaoweiChen/CPP-Move-Semantics}
\end{itemize}