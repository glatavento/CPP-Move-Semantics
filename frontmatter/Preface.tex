\begin{flushright}
	\zihao{0} 前言
\end{flushright}

目前为止,每当教授移动语义时,我都会说,“必须有人写一本关于移动的书,”通常的回答是:“那必须的的!你来吧!”。现在,我终于做到了。

和往常一样,写一本关于C++的书时,我都会对要介绍的知识、要阐明的情况和要描述的结果感到惊讶。是时候出一本关于Move语义的书了,要覆盖了从C++11到C++20的所有版本。我在这个过程中学到了很多,相信你也一样。 

\subsubsection{本书就是实验}

本书完成了实验的两个方面:
\begin{itemize}
	\item 写一本有深度的书,会涉及复杂的核心语言特性,还没有核心语言专家的直接帮助。不过,我可以通过问问题来写这本书。
	\item 我自己在Leanpub上出版这本书,并按需印刷。这本书是逐步完成的,有了改进,就会发布一个新的版本。
\end{itemize}

好的方面是:
\begin{itemize}
	\item 可以从经验丰富的编程人员那里了解语言特性——他们体会到某个特性可能造成的痛苦,并提出相关问题,以便能够激励和解释设计,以及其在实践中编程的结果。
	\item 当我还在写作时,可以阅读我的移动语义经验,并进行借鉴。
	\item 这本书和所有的读者都能从您的早期反馈中受益。
\end{itemize}

这意味着你也是实验的一部分。所以,请帮助我:对本书的缺陷、错误、未解释清楚的功能给予反馈,这样大家都能从这些改进中获益。

\subsubsection{致谢}

首先,我要感谢C++社区,是你们使这本书成为可能。移动语义的功能的设计,有用的反馈,他们的好奇心是语言进化的基础。特别感谢那些告诉我和解释的所有问题的人们,以及给我的反馈,谢谢你们。

我要特别感谢每一个为这本书或审阅草稿并提供有价值的反馈的人。这些评论大大提高了这本书的质量,再次证明了好东西需要许多“聪明人”的投入。目前为止(这个列表还在增长),感谢你们:Javier Estrada, Howard Hinnant, Klaus Iglberger, Daniel Kr ugler, Marc Mutz, Aleksandr Solovev (alexolut), Peter Sommerlad和Tony Van Eerd。

此外,我还要感谢C++社区和C++标准委员会的每一个人。除了所有涉及添加新语言和库功能的工作外,这些专家还花了很多很多的时间和我解释和讨论他们的工作,他们很有耐心,也很有热情。

特别感谢LaTeX社区提供的文本系统,感谢Frank Mittelbach解决了我的 \LaTeX{} 问题。

最后,非常感谢本书校对,Tracey Duffy,她做了大量的工作,把我的“德式英语”翻译成地道的英语。


















